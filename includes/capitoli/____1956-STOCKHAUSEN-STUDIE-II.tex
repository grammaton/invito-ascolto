%!TEX TS-program = xelatex
%!TEX encoding = UTF-8 Unicode
%!TEX root = ../2017-GS-COME01_xelatex_A4_UTF8.tex

%-------------------------------------------------------------------------------------------------------------------------------------------
%%\twocolumn
%%
%%\chapter{Estratti Bibliografici.}
%%\section*{da \textit{Musica ex Machina\footnote{Musica Ex Machina Ð Prieberg (pag. 170 - 179)}}}
%
%
[\ldots] Il suono sinusoidale  un fenomeno completamente nuovo nella musica. Si intende un suono senza armonici, cio il suono nudo, che forma una vibrazione sinusoidale. é un prodotto dei vibratori graduati elettronici, con i quali di solito vengono prodotti nelle stazioni radio il segnale orario, il diapason per i musicisti e Ð prima e dopo l'orario giornaliero di trasmissione Ð le frequenze pilota. [\ldots] é veramente senza luogo, una vibrazione isolata, che nasce ed  amplificata nella valvola elettronica e diventa udibile solo nell'altoparlante. Il suono sinusoidale si sottrae a  ogni definizione all'infuori di quella con il numero di \emph{Hertz} delle sue vibrazioni. Girando l'interruttore del condensatore del generatore elettronico lo si pu˜ spostare in qualsiasi punto dell'intera sfera uditiva. Il suo suono, secondo la precisa descrizione di Herbert Eimert, 
\begin{quotation}
\begin{sf}
\begin{small}
\noindent neutrale e giˆ abbastanza forte a un'intensitˆ media, perci˜ non  affatto smorto e insussistente. Siccome manca degli armonici caratteristici, non ha alcun timbro ben definito. Il suo contrassegno principale  la diretta immediatezza del suono. Dipende dalla sua natura elettrica il fatto che esso risuona come una corrente uniforme, rigido e non modulato.
\end{small}
\end{sf} 
\end{quotation}

\noindent [\ldots] Suoni sinusoidali possono essere sovrapposti in qualsiasi numero e frequenza, sia armonicamente, di modo che si ottiene un suono con un corteggio di armonici naturale, sia non armonicamente, di modo che nasce una cosiddetta <<~mescolanza di suoni~>>. [\ldots] L'inizio del suono coincide con l'innesto, e con il disinnesto s'interrompe immediatamente. Perci˜ il carattere del suono deve essere aggiunto alla composizione e per fare questo si offrono le pi varie possibilitˆ. Con le forbici il compositore taglia via dal nastro inciso non solo la lunghezza ma anche le cosiddette <<~curve di inviluppo~>>, le caratteristiche forme dei suoi elementi acustici. [\ldots] Inoltre egli pu˜ servirsi di un apparecchio di risonanza o della riverberazione che si trova in ogni stazione radio. Se la formazione lo soddisfa, inizia allora il montaggio dei molti piccoli ritagli di nastro nella disposizione ritmica desiderata.

Complicatissimi passaggi dinamici nascono sotto un costante controllo con il centimetro. Siccome il compositore conosce la velocitˆ del nastro, di solito 76,2 o 38,1 centimetri al secondo, misurando la lunghezza di pause e suoni egli pu˜ produrre ritmi <<~irrazionali~>> precisissimi, mentre ogni compositore di musica strumentale fallirebbe senza speranza in quest'impresa.

%[\ldots] L'uomo irruppe nel regno della macchina, nella sfera a essa peculiare della trasformazione dell'energia, che in un primo tempo serviva soltanto a facilitare e accelerare il suo lavoro.

[\ldots] Esistono diverse possibilitˆ di fissare la musica elettronica in qualcosa di simile a uno <<~spartito~>>. Una <<~partitura~>> elettronica, lo \emph{Studio II} di Karlheinz Stockhausen,  giˆ stampata. Un progetto di costruzione, un disegno tecnico per cos’ dire, documento unico di una musica dell'avvenire. Al primo sguardo si presenta come un disegno affascinante di vari rettangoli e triangoli.

L'idea della notazione, cio la distribuzione del tempo in senso orizzontale da sinistra a  destra, e la collocazione delle note e dei suoni in senso verticale, dal basso in alto, rimane inalterata. Invece  nuova l'annotazione assoluta. Le note musicali si limitano a dare un sistema di riferimento. Esse sono relative. Il loro valore assoluto dipende da una convenzione non obbligatoria, dall'altezza relativa del diapason. Un'esecuzione pi o meno autentica Ð anche per quel che riguarda l'altezza di suono Ð riesce unicamente se si ha una precisa conoscenza di questa convenzione. Invece la musica elettronica  incisa su nastro conformemente alle idee del compositore. L'interpretazione non  nŽ necessaria nŽ possibile. Perci˜ la rappresentazione grafica dell'opera elettronica deve essere vera e assolutamente precisa: il tecnico dello studio lavora su  di essa. Molte novitˆ saltano agli occhi. L'altezza del suono si misura in \emph{Hertz}, cio in vibrazioni al secondo; l'intensitˆ si esprime in \emph{Decibel}, cio nei gradi di un aumento del 26 per cento circa dell'energia sonora, che vengono ancora percepiti dall'orecchio Ð non pi con indicazioni cos“ vaghe come forte e pianissimo; per la velocitˆ si evitano le arbitrarie indicazioni andante, presto o largo, si calcola secondo i centimetri del nastro che scorre. Al posto dell'approssimazione interpretativa  subentrata matematica esattezza. La <<~partitura~>> elettronica di Stockhausen corrisponde alle premesse di un caso compositivo speciale: le centonovantatre mescolanze di suoni, di cinque suoni sinusoidali ciascuno, che egli usa, non sono montati singolarmente ma risultano dalla relazione meccanica dei suoni sinusoidali sonati uno dopo l'altro nella riverberazione. Siccome si tratta esclusivamente di  suoni di uguale intensitˆ e con intervalli costanti, ne consegue subito che una simile annotazione semplificata pu˜ valere soltanto  per questa composizione elettronica.

Se si esamina ora nei particolari, si vede che la parte superiore della <<~partitura~>> Ð un po' pi della metˆ Ð  costituita da un sistema di linee che comprende lo spazio da 100 fino a 17200 \emph{Hertz} sfruttato da Stockhausen nello \textit{Studio II}. Rettangoli di varie forme simbolizzano blocchi di cinque suoni sinusoidali ciascuno, cio mescolanze di suoni. Sotto questa parte la lunghezza dei blocchi  riportata ancora una volta su di una scala in centimetri di nastro. 76,2 centimetri corrispondono alla velocitˆ allora usuale del nastro. Pi sotto ancora vi  un secondo pi sottile sistema di linee per la dinamica da 40\footnote{in realtˆ  da Ð40 a 0 \emph{dB}} fino a 0 decibel. L'altezza del grafico in questo sistema indica l'intensitˆ corrispondente. Le sue forme determinano i contorni, cio le curve di inviluppo delle mescolanze dei suoni. In alcuni punti dei due sistemi mancano figure, il che significa pausa. Anche la durata  visibile dal numero nella scala delle lunghezze. Una pagina della partitura rappresenta all'incirca sei secondi di musica.

I primi sette pezzi elettronici dello studio di Colonia [\ldots] Sono come uno scenario acustico, che provoca senza dubbio eccitamenti molto violenti i quali, pur non causando uno shock, opprimono tuttavia in modo quasi insopportabile il sistema nervoso.

%\section*{da \textit{La Musica Elettronica\footnote{La Musica Elettronica Ð Pousseur  (pag. 65 - 68)}}}
[\ldots] L'articolazione e l'esatto metodo di produzione del materiale sono stati interamente esposti nella prefazione alla ÒpartituraÓ, una delle rare rappresentazioni di musica elettronica pubblicate: si potrˆ rimandare ad essa. In effetti, in essa 
si ottiene una fusione molto maggiore degli ÒelementiÓ all'interno dei Òsuoni complessiÓ. Essa  dovuta, in primo luogo, alla loro Òuguaglianza dinamicaÓ ed alla complessitˆ molto pi grande dei loro rapporti armonici (ricordiamo che nel \textit{Primo Studio} non si trattava di spettri ÒarmoniciÓ propriamente detti, centrati su un fondamentale unico; almeno tutti gli intervalli facenti parte di un blocco erano degli intervalli giusti, degli intervalli semplici e ÒtrasparentiÓ). Essa  
anche dovuta, per alcuni di essi (per i pi ÒagglomerantiÓ), al serrarsi di questi intervalli costitutivi che provoca (non soltanto per la percezione, ma, si potrebbe dire, oggettivamente) la distruzione reciproca delle periodicitˆ individuali e (come giˆ il cromatismo simultaneo nelle musiche strumentali di cui abbiamo parlato nel primo capitolo) genera dei veri rumori (molto controllati, tuttavia). Inoltre, l'utilizzazione della camera a eco (ÒnaturaleÓ) come uno degli stessi mezzi della produzione sonora (metodo che reintroduce fin da allora un elemento non elettronico e per questo non strettamente controllato) rinforza ancora la fusione delle componenti (valorizzando i legami interferenziali) e conferisce ai risultati un'unitˆ supplementare, dovuta al colore proprio di cui riveste in un certo senso tutto ci˜ che 
passa attraverso di essa. 

Infine, l'attacco simultaneo di diversi blocchi di questa specie da cui si ritagliano eventualmente le regioni armoniche (e la cui ulteriore evoluzione divergente dˆ, per contrasto, conferma), ed il fatto che Stockhausen abbia superato su questi attacchi di un poco le soglie di registrazione del nastro magnetico (di passare nella zona ÒrossaÓ dei potenziometri: fino a + 6dB), producendo, con la distorsione che ne deriva, dei veri transitori, paragonabili agli attacchi di certi strumenti (a percussione), contribuiva ad ottenere dei fenomeni sonori molto pi unitari, la cui unitˆ era giˆ caratterizzata da un certo tasso di evoluzione interna, e li rendeva capaci di sopportare meglio il paragone con il carattere ÒorganicoÓ dei fenomeni naturali. Certo, a questo si univa una (molto relativa, ma innegabile nel rigore della prospettiva iniziale) perdita di controllo. Tuttavia le immediate conclusioni che se ne sarebbero potute tirare, le conseguenze che questo 
avrebbe avuto, nel senso di un certo ammorbidimento dei principi di realizzazione (e in primo luogo di concezione) non erano il solo insegnamento che se ne potesse trarre: Stockhausen (e coloro che seguivano da vicino le sue esperienze cominciando eventualmente a dedicarsi ad esperienze parallele) aveva appreso qui ogni sorta di nozioni precise sulla struttura dei fatti sonori, sulla possibilitˆ di continuare a ricercarne il controllo integrale, anche se questo dovesse durare per un periodo abbastanza lungo e passare attraverso tappe apparentemente in contraddizione. 

In effetti, fin dal compimento del \textit{Primo Studio} di Stockhausen, altri compositori erano stati invitati a lavorare temporeneemente allo studio di Colonia ed a realizzarvi una composizione per forza di cose modesta. Cosi, alla fine del 1954, una prima esecuzione dei lavori (che occupavano la metˆ di un concerto) potŽ venir proposta al pubblico (l'altra parte comprendeva esecuzioni di nuova musica americana da parte di John Cage e di David Tudor). Ad eccezione di qualche dettaglio, tuttavia, nessuno di questi lavori apportava nulla di fondamentalmente nuovo rispetto alle realizzazioni di Stockhausen.

[\ldots] fenomeni paragonabili fino a un certo punto agli aggregati pi indivisibili del \textit{Secondo Studio} di Stockhausen, potevano essere ottenuti filtrando un fenomeno elettronico il meno periodico, il pi disordinato possibile, la cui applicazione acustica si chiama Òrumore biancoÓ. Ed infine alcuni momenti dell'una o dell'altra composizione particolarmente movimentati, particolarmente ÒmicrostrutturatiÓ provavano (soprattutto se accelerati ancora, cosa di cui si poteva fare esperienza quotidiana durante il lavoro di studio) che si potevano raggiungere unitˆ di una nuova specie, la cui instabilitˆ, la cui mobilitˆ, sarebbe una delle caratteristiche principali, che le contrapporrebbe dunque in maniera molto netta alla maggior parte dei suoni strumentali (analoghe solo alle misture a un tempo molto dense e molto rapide, come ne esistevano giˆ, per esempio nella seconda delle prime \textit{Structures} per due pianoforti di Boulez, o nei \textit{Kontrapunkte} di Stockhausen). Prescindendo dalla perdita di controllo che il tentativo di sistematizzare dei fenomeni di questo tipo poteva rappresentare (e che potrebbe, per lo meno provvisoriamente, essere compensato da criteri di determinazione statistica) era proprio questo l'effetto di una di quelle Òimmaginazioni concreteÓ di cui dicevamo, che furono all'origine della nascita della musica seriale generalizzata, immaginazione che ci sforziamo dunque, intravedendola, di rendere attuale con un massimo di efficacia, ad un tempo ÒespressivaÓ (cio piuttosto ÒqualitativaÓ) e strutturale (o pi ÒquantitativaÓ). Sembrava che ci fossero delle vie per condurre a questo senza passare per il missaggio ed il montaggio di elementi sinusoidali (operazione naturalmente fastidiosa nel caso in cui si vogliano realizzare dei fenomeni formalmente antinomici rispetto alla sinusoide Ð e del resto poco efficaci a causa dei Òrumori di fondo,Ó nel senso molto generale della parola, che la molteplicitˆ delle operazioni introduce ed accumula). Le sinusoidi non rappresentarono dunque pi che uno degli estremi di un campo di cui gli altri due ÒpoliÓ simbolici sembrano essere: l'onda periodica ÒangolareÓ cio il meno sinusoidale possibile, definibile in termini di ÒimpulsoÓ (dente di sega, ÒagoÓ o rettangolo) ed il Òrumore biancoÓ o processo vibratorio il meno periodico possibile.

Questo allargamento delte ÒriserveÓ materiali a cui poteva attingere il compositore apriva ad un tratto alla musica elettronica uno spazio figurativo molto pid ricco, molto pi duttile. Se i compositori sapessero rivelarsi sufficientemente attenti, la luce momentanea (ed in alcuni casi del tutto relativa) del principio del controllo integrale, potrebbe non essere che una specie di astuzia per permettere l'appropriazione e la progressiva realizzazione di questi materiali sonori in diverse tappe. Tuttavia la prima di esse sarebbe una tappa in cui l'accento sarebbe posto pi sovente, perlomeno per una parte dei compositori (e non necessariamente per i meno ÒprospetticiÓ), sull'aspetto qualitativo, spontaneo, e quindi sulla generazione relativamente empirica, talvolta davvero improvvisata, delle nuove sonoritˆ).

[\ldots]
%
%%\section*{da \textit{Musica Espansa\footnote{Musica Espansa Ð Galante, Sani Ð Le Sfere}}}
%Gran parte dell'esperienza seriale elettronica maturata a Colonia  sorretta dall'operare teorico a compositivo di Stockhausen, dalla sua estrema fiducia nell'analisi cognitiva e da una metodologia scientifica di lavoro. Nel periodo dal 1953 al 1954, egli realizza alla WDR i pezzi elettronici \emph{Studie I} e \emph{II}, che rimangono tra i progetti musicali pi rappresentativi di quella stagione musicale. Un condensato di riflessione teorica e di tecnica realizzativa, ma anche di scontro compositivo con il materiale e con i problemi della percezione musicale\footnote{(\emph{ndr} Ð nota presente nel testo originale) Ð Lo \emph{Studie II}  inoltre il primo esempio di musica elettronica rappresentata da una partitura, un tentativo riuscito di creare una mediazione tra gesto formale e ascolto, mediante una descrizione grafica puntuale dello spazio sonoro: ampiezze, durate, misture frequenziali, inviluppi, densitˆ verticale e movimento delle sequenze nel tempo. Dal punto di vista della grafia musicale nella musica elettronica degli anni Cinquanta, esistono altri esempi importanti, tra cui \emph{Incontri di fasce sonore} (1956) di F. Evangelisti, e \emph{Artikulation} (1958) di G. Ligeti. Entrambe hanno avuto una stesura che si avvicina a delle partiture di ascolto. Altre musiche concrete o elettroniche dispongono di partiture che vanno intese come partiture di lavoro, tra queste: \emph{Timbre-Dures} (1952) di O. Messiaen, \emph{Essay} (1957) di M. Koenig. Il problema di una grafia della musica elettronica  anche contraddittorio per il significato stesso di partitura, cio di mezzo simbolico non astratto ma finalizzato per l'esecuzione musicale. Il nastro magnetico  di per sŽ la partitura.}. 
%
%Se lo \emph{Studie I} pu˜ essere considerato una prova generale importante di progettazione musicale nel laboratorio della serialitˆ elettronica, lo \emph{Studie II} rivela un approfondimento compositivo indirizzato a ricercare una maggiore caratterizzazione interna dei materiali elettronici, a un maggiore contrasto tra le diverse tipologie delle misture sonore. In particolare, caratterizzando il pezzo con un uso delle strutture frequenziali molto pi ricco e complesso, caricato da una bassa armonicitˆ, in cui prevale alla fine una risultante timbrica assimilabile a quella ottenuta con rumori variamente filtrati e inviluppati. La contrapposizione tra i diversi transitori di attacco e tra le dinamiche  molto pi netta; il fortissimo si avvale anche di una controllata saturazione dinamica del magnetofono, che conferisce di conseguenza una maggiore ÒcoloraturaÓ che, come sottolinea Henri Pousseur, Ò contribuiva a ottenere dei fenomeni sonori molto pi unitari, la cui unitˆ era giˆ caratterizzata da un certo tasso di evoluzione interna, che li rendeva capaci di sopportare meglio il paragone con il carattere organico dei fenomeni naturaliÓ\footnote{(\emph{ndr} Ð nota presente nel testo originale) Ð \emph{La musica elettronica}, Feltrinelli, Milano 1976}. 
%
%é utile portare l'attenzione sulla definizione di ÒorganicoÓ in contrapposizione al non organico che Pousseur utilizza nella descrizione del modo di suonare di certi blocchi sonori dello \textit{Studie II}. Tale osservazione mette in luce uno degli aspetti di maggiore criticitˆ della musica elettronica seriale, dovuta all'immobilitˆ interna del suono. alla mancanza di microarticolazione della materia che conferisce viceversa una stimolazione percettiva e un interesse nell'ascolto e che ci permette di parlare di timbro e di agglomerati sinusoidali, cosa che Stockhausen aveva rilevato.
%
%Da un lato, il permutare e il combinare continuo dei parametri acustici delle Òmisture sonoreÓ non facilita l'articolazione del tessuto sonoro, a causa del prodursi di una continua indifferenziazione, dovuta anche all'esiguo numero di parziali che compongono in genere le misture; dall'altra le difficoltˆ intrinseche nella realizzazione di una vera e propria sintesi additiva del timbro. Quindi il Òcomporre il suonoÓ con i mezzi tecnologici dell'epoca, conduce ad accentuare volutamente soluzioni pi dirette. Da qui l'introduzione di accorgimenti elaborativi estranei, come il passaggio delle misture nella Òcamera di riverberazioneÓ, con lo scopo di conferire al risultato contorni Òdi disturboÓ non prevedibili e quindi aleatori, in grado di dare respiro e circolazione sanguigna ai blocchi di suono.
%
%[\ldots]
%
%In generale l'ascolto delle opere realizzate nello Studio di Colonia rivela una difficoltˆ nell'uscire da un'omologazione timbrica e da una economia dei materiali che non permette di ottenere una maggiore flessibilitˆ linguistica.
%
%é individuabile un interesse primario rivolto soprattutto al sistema delle frequenze, che guida il procedere compositivo anche in un simile contesto sperimentale; Stockhausen, per esempio, utilizza in \emph{Studie II} un rapporto di 5:1 suddiviso in 25 parti. Probabilmente ha ragione Franco Evangelisti quando sottolinea la necessitˆ di ricorrere a rapporti di distanza, nello spazio frequenziale, espressi da numeri non interi, come condizione per uscire da certi vincoli di periodicitˆ nella generazione spettrale dei suoni e conquistare realmente un diverso mondo sonoro. La sua composizione \emph{Incontri di fasce sonore} (1956)  l'ultimissimo esempio di una musica elettronica che parte da un'idea costruttiva del timbro secondo il modello delle misture sonore sinusoidali, anche se esistono nel pezzo altre soluzioni a sostegno di una maggiore ricchezza di materiale disponibile. Scrive Evangelisti a proposito dell'organizzazione delle misture sonore presenti nella sua opera:
%\begin{quotation}
%{\small \textsf{La suddivisione dello spazio sonoro e delle sue durate  stata concepita in base ai rapporti psicofisici e in funzione dei parametri degli stessi. La scala delle grequenze  stata suddivisa in 91 gradi a partire dalla frequenza pi bassa di 87 Hz alla pi alta di 11ú950 Hz. Il problema fondamentale  stato qello di non stabilire rapporti di armonia di nessun genere, espressi mediante numeri razionali interi, come per esempio nella scala temperata il rapporto di 2:1, o come nello studio di k. Stockhausen il rapporto di 5:1. [\ldots]}}
%\end{quotation}
%
%La questione non risolta della mobilitˆ interna del suono e della diversificazione dei materiali  un problema di tale complessitˆ che gli assiomi iniziali vengono ben presto rivisitati e frantumata cos“ l'estetica dell'elettronica pura. 
%\newpage
%
%%\chapter{Traduzione delle note di realizzazione}
%%
%\section*{Premessa}
%Le sezioni che seguono sono la traduzione dall'inglese, da me effettuata, della indicazioni in partitura, edita dalla Universal Edition.
%
%Tutte le indicazioni di frequenza espresse nella versione inglese in \emph{c.p.s.} (cicles per second) sono state tradotte in \emph{Hz}, l'unitˆ di misura usata per convenzione, come anche nella versione tedesca.
%
%\section{Introduzione}
%La partitura di Studio II  la prima partitura, di Musica Elettronica, ad essere stata pubblicata. Fornisce al tecnico tutte le informazioni necessarie alla realizzazione del lavoro, e pu˜ essere usata da musicisti e amanti di musica come partitura di studio Ð preferibilmente in relazione all'ascolto della musica.\\
%La prima realizzazione fu fatta nel 1954 nello ÒStudio per la Musica ElettronicaÓ della Radio di Colonia (WDR).\\
%Per questo nuovo tipo di composizione vennero sviluppate adeguate forme di notazione. Le seguenti spiegazioni della partitura nascono dalle procedure di lavoro utilizzate per questa composizione.
%
%\section{Altezza}
%é stata scelta una scala di 81 gradi che parte da 100 \emph{Hz} e sale con l'intervallo costante di $\sqrt[25] 5$. Le frequenze sono state arrotondate ai valori ottenibili con gli oscillatori RC utilizzati\footnote{L'arrotondamento avviene per troncamento poichŽ il tipo di oscillatori non permetteva frequenze con valori decimali. L'oscillatore RC, della categoria dei sinusoidali,  principalmente composto da \emph{Resistori} e \emph{Capacitori}. (\emph{ndr}).}. Vengono composte cinque note sinusoidali a intervalli costanti per ogni mistura (vedi ÒRealizzazioneÓ). Sono usate cinque differenti misture di note, con intervalli costanti rispettivamente di 1, 2, 3, 4 o 5 volte $\sqrt[25] 5$. Le risultanti misture di note numerate da 1 a 193 costituiscono il materiale sonoro per questo studio (in \emph{Hz}). Vedi tabella.
%
%
%
%%\newpage
%%\thispagestyle{empty}
%%\begin{landscape}
%%Tabella Misture
%%\vfill
%%\begin{table}[hb]
%%\begin{sf}
%%\tiny
%%
%%\makebox[\linewidth]{% Inizio scatola 
%%
%%\begin{tabular}{rrrrrr|rrrrrrr|rrrrrrr|rrrrrrr|rrrrrrr|rrrrrrr|rrrrrrr|rrrrrrr|rrrrrrr}
%%
%%\textbf{1} & 100 & 107 & 114 & 121 & 129 & & \textbf{26} & 138 & 147 & 157 & 167 & 178 & & \textbf{47} & 190 & 203 & 217 & 231 & 246 & & \textbf{68} & 263 & 280 & 299 & 319 & 340 & & \textbf{89} & 362 & 386 & 412 & 440 & 469 & & \textbf{110} & 500 & 533 & 569 & 607 & 647 & & \textbf{131} & 690 & 736 & 785 & 837 & 893 & & \textbf{152} & 952 & 1010 & 1080 & 1150 & 1230 & & \textbf{173} & 1310 & 1400 & 1490 & 1590 & 1700 \\ 
%%\textbf{2} & 107 & 114 & 121 & 129 & 138 & & 27 & 147 & 157 & 167 & 178 & 190 & & 48 & 203 & 217 & 231 & 246 & 263 & & 69 & 280 & 299 & 319 & 340 & 362 & & 90 & 386 & 412 & 440 & 469 & 500 & & 111 & 533 & 569 & 607 & 647 & 690 &  & 132 & 736 & 785 & 837 & 893 & 952 &  & 153 & 1010 & 1080 & 1150 & 1230 & 1310 &  & 174 & 1400 & 1490 & 1590 & 1700 & 1810 \\ 
%%\textbf{3} & 114 & 121 & 129 & 138 & 147 &  & 28 & 157 & 167 & 178 & 190 & 203 &  & 49 & 217 & 231 & 246 & 263 & 280 &  & 70 & 299 & 319 & 340 & 362 & 386 &  & 91 & 412 & 440 & 469 & 500 & 533 &  & 112 & 569 & 607 & 647 & 690 & 736 &  & 133 & 785 & 837 & 893 & 952 & 1010 &  & 154 & 1080 & 1150 & 1230 & 1310 & 1400 &  & 175 & 1490 & 1590 & 1700 & 1810 & 1930 \\ 
%%\textbf{4} & 121 & 129 & 138 & 147 & 157 &  & 29 & 167 & 178 & 190 & 203 & 217 &  & 50 & 231 & 246 & 263 & 280 & 299 &  & 71 & 319 & 340 & 362 & 386 & 412 &  & 92 & 440 & 469 & 500 & 533 & 569 &  & 113 & 607 & 647 & 690 & 736 & 785 &  & 134 & 837 & 893 & 952 & 1010 & 1080 &  & 155 & 1150 & 1230 & 1310 & 1400 & 1490 &  & 176 & 1590 & 1700 & 1810 & 1930 & 2060 \\ 
%%\textbf{5} & 129 & 138 & 147 & 157 & 167 &  & 30 & 178 & 190 & 203 & 217 & 231 &  & 51 & 246 & 263 & 280 & 299 & 319 &  & 72 & 340 & 362 & 386 & 412 & 440 &  & 93 & 469 & 500 & 533 & 569 & 607 &  & 114 & 647 & 690 & 736 & 785 & 837 &  & 135 & 893 & 952 & 1010 & 1080 & 1150 &  & 156 & 1230 & 1310 & 1400 & 1490 & 1590 &  & 177 & 1700 & 1810 & 1930 & 2060 & 2200 \\ 
%%
%%&  &  &  &  &  &  &  &  &  &  &  &  &  &  &  &  &  &  &  &  &  &  &  &  &  &  &  &  &  &  &  &  &  &  &  &  &  &  &  &  &  &  &  &  &  &  &  &  &  &  &  &  &  &  &  &  &  &  &  &  & \\
%%
%% \textbf{6} & 100 & 114 & 129 & 147 & 167 &  & 31 & 138 & 157 & 178 & 203 & 231 &  & 52 & 190 & 217 & 246 & 280 & 319 &  & 73 & 263 & 299 & 340 & 386 & 440 &  & 94 & 362 & 412 & 469 & 533 & 607 &  & 115 & 500 & 569 & 647 & 736 & 837 &  & 136 & 690 & 785 & 893 & 1010 & 1150 &  & 157 & 952 & 1080 & 1230 & 1400 & 1590 &  & 178 & 1310 & 1490 & 1700 & 1930 & 2200 \\ 
%%\textbf{7} & 114 & 129 & 147 & 167 & 190 &  & 32 & 157 & 178 & 203 & 231 & 263 &  & 53 & 217 & 246 & 280 & 319 & 362 &  & 74 & 299 & 340 & 386 & 440 & 500 &  & 95 & 412 & 469 & 533 & 607 & 690 &  & 116 & 569 & 647 & 736 & 837 & 952 &  & 137 & 785 & 893 & 1010 & 1150 & 1310 &  & 158 & 1080 & 1230 & 1400 & 1590 & 1810 &  & 179 & 1490 & 1700 & 1930 & 2200 & 2500 \\ 
%%\textbf{8} & 129 & 147 & 167 & 190 & 217 &  & 33 & 178 & 203 & 231 & 263 & 299 &  & 54 & 246 & 280 & 319 & 362 & 412 &  & 75 & 340 & 386 & 440 & 500 & 569 &  & 96 & 469 & 533 & 607 & 690 & 785 &  & 117 & 647 & 736 & 837 & 952 & 1080 &  & 138 & 893 & 1010 & 1150 & 1310 & 1490 &  & 159 & 1230 & 1400 & 1590 & 1810 & 2060 &  & 180 & 1700 & 1930 & 2200 & 2500 & 2840 \\ 
%%\textbf{9} & 147 & 167 & 190 & 217 & 246 &  & 34 & 203 & 231 & 263 & 299 & 340 &  & 55 & 280 & 319 & 362 & 412 & 469 &  & 76 & 386 & 440 & 500 & 569 & 647 &  & 97 & 533 & 607 & 690 & 785 & 893 &  & 118 & 736 & 837 & 952 & 1080 & 1230 &  & 139 & 1010 & 1150 & 1310 & 1490 & 1700 &  & 160 & 1400 & 1590 & 1810 & 2060 & 2340 &  & 181 & 1930 & 2200 & 2500 & 2840 & 3230 \\ 
%%\textbf{10} & 167 & 190 & 217 & 246 & 280 &  & 35 & 231 & 263 & 299 & 340 & 386 &  & 56 & 319 & 362 & 412 & 469 & 533 &  & 77 & 440 & 500 & 569 & 647 & 736 &  & 98 & 607 & 690 & 785 & 893 & 1010 &  & 119 & 837 & 952 & 1080 & 1230 & 1400 &  & 140 & 1150 & 1310 & 1490 & 1700 & 1930 &  & 161 & 1590 & 1810 & 2060 & 2340 & 2670 &  & 182 & 2200 & 2500 & 2840 & 3230 & 3680 \\ 
%%
%%&  &  &  &  &  &  &  &  &  &  &  &  &  &  &  &  &  &  &  &  &  &  &  &  &  &  &  &  &  &  &  &  &  &  &  &  &  &  &  &  &  &  &  &  &  &  &  &  &  &  &  &  &  &  &  &  &  &  &  &  & \\
%%
%%\textbf{11} & 100 & 121 & 147 & 178 & 217 &  & 36 & 148 & 167 & 203 & 246 & 299 &  & 57 & 290 & 231 & 280 & 340 & 412 &  & 78 & 263 & 319 & 386 & 469 & 569 &  & 99 & 362 & 440 & 533 & 647 & 785 &  & 120 & 500 & 607 & 736 & 893 & 1080 &  & 141 & 690 & 837 & 1010 & 1230 & 1490 &  & 162 & 952 & 1150 & 1400 & 1700 & 2060 &  & 183 & 1310 & 1590 & 1930 & 2340 & 2540 \\ 
%%\textbf{12} & 121 & 147 & 178 & 217 & 263 &  & 37 & 167 & 203 & 246 & 299 & 362 &  & 58 & 231 & 280 & 340 & 412 & 500 &  & 79 & 319 & 386 & 469 & 569 & 690 &  & 100 & 440 & 533 & 647 & 785 & 952 &  & 121 & 607 & 736 & 893 & 1080 & 1310 &  & 142 & 837 & 1010 & 1230 & 1490 & 1810 &  & 163 & 1150 & 1400 & 1700 & 2060 & 2500 &  & 184 & 1590 & 1930 & 2340 & 2840 & 3450 \\ 
%%\textbf{13} & 147 & 178 & 217 & 263 & 319 &  & 38 & 203 & 246 & 299 & 362 & 440 &  & 59 & 280 & 340 & 412 & 500 & 607 &  & 80 & 386 & 469 & 569 & 690 & 837 &  & 101 & 533 & 647 & 785 & 952 & 1150 &  & 122 & 736 & 893 & 1050 & 1310 & 1590 &  & 143 & 1010 & 1230 & 1490 & 1810 & 2200 &  & 164 & 1400 & 1700 & 2060 & 2500 & 3030 &  & 185 & 1930 & 2340 & 2840 & 3450 & 4180 \\ 
%%\textbf{14} & 178 & 217 & 263 & 319 & 386 &  & 39 & 246 & 299 & 362 & 440 & 533 &  & 60 & 340 & 412 & 500 & 667 & 736 &  & 81 & 469 & 569 & 690 & 837 & 1010 &  & 102 & 647 & 785 & 952 & 1150 & 1400 &  & 123 & 893 & 1080 & 1310 & 1590 & 1930 &  & 144 & 1230 & 1490 & 1810 & 2200 & 2670 &  & 165 & 1700 & 2060 & 2500 & 3030 & 3680 &  & 186 & 2340 & 2840 & 3450 & 4180 & 5080 \\ 
%%\textbf{15} & 217 & 263 & 319 & 386 & 469 &  & 40 & 299 & 362 & 330 & 533 & 647 &  & 61 & 412 & 500 & 607 & 736 & 893 &  & 82 & 569 & 690 & 837 & 1010 & 1230 &  & 103 & 785 & 952 & 1150 & 1400 & 1700 &  & 124 & 1080 & 1310 & 1590 & 1930 & 2340 &  & 145 & 1490 & 1810 & 2200 & 2670 & 3230 &  & 166 & 2060 & 2500 & 3030 & 3680 & 4460 &  & 187 & 2840 & 3450 & 4180 & 5080 & 6160 \\ 
%%
%%&  &  &  &  &  &  &  &  &  &  &  &  &  &  &  &  &  &  &  &  &  &  &  &  &  &  &  &  &  &  &  &  &  &  &  &  &  &  &  &  &  &  &  &  &  &  &  &  &  &  &  &  &  &  &  &  &  &  &  &  & \\
%%
%%\textbf{16} & 100 & 129 & 167 & 217 & 280 &  & 41 & 138 & 178 & 231 & 299 & 386 &  & 62 & 190 & 246 & 319 & 412 & 533 &  & 83 & 263 & 340 & 440 & 569 & 736 &  & 104 & 362 & 469 & 607 & 785 & 1010 &  & 125 & 500 & 647 & 837 & 1080 & 1400 &  & 146 & 690 & 893 & 1150 & 1490 & 1930 &  & 167 & 952 & 1230 & 1590 & 2060 & 2670 &  & 188 & 1310 & 1700 & 2200 & 2840 & 3680 \\ 
%%\textbf{17} & 129 & 167 & 217 & 280 & 362 &  & 42 & 178 & 231 & 299 & 386 & 500 &  & 63 & 246 & 319 & 412 & 533 & 690 &  & 84 & 340 & 440 & 569 & 736 & 952 &  & 105 & 469 & 607 & 785 & 1010 & 1310 &  & 126 & 647 & 837 & 1080 & 1400 & 1810 &  & 147 & 893 & 1150 & 1490 & 1930 & 2500 &  & 168 & 1230 & 1590 & 2060 & 2670 & 3450 &  & 189 & 1700 & 2200 & 2840 & 3680 & 4760 \\ 
%%\textbf{18} & 167 & 217 & 280 & 362 & 469 &  & 43 & 231 & 199 & 386 & 500 & 647 &  & 64 & 319 & 412 & 533 & 690 & 893 &  & 85 & 440 & 569 & 736 & 952 & 1230 &  & 106 & 607 & 785 & 1010 & 1310 & 1700 &  & 127 & 837 & 1080 & 1400 & 1810 & 2340 &  & 148 & 1150 & 1490 & 1930 & 2500 & 3230 &  & 169 & 1590 & 2060 & 2670 & 3450 & 4460 &  & 190 & 2200 & 2840 & 3680 & 4760 & 6160 \\ 
%%\textbf{19} & 217 & 280 & 362 & 469 & 607 &  & 44 & 299 & 286 & 500 & 647 & 837 &  & 65 & 412 & 533 & 690 & 893 & 1150 &  & 86 & 569 & 736 & 952 & 1230 & 1590 &  & 107 & 785 & 1010 & 1310 & 1700 & 2200 &  & 128 & 1080 & 1400 & 1810 & 2340 & 3030 &  & 149 & 1490 & 1930 & 2500 & 3230 & 4180 &  & 170 & 2060 & 2670 & 3450 & 4460 & 5770 &  & 191 & 2840 & 3680 & 4760 & 6160 & 7970 \\ 
%%\textbf{20} & 280 & 362 & 469 & 607 & 785 &  & 45 & 386 & 500 & 647 & 837 & 1080 &  & 66 & 533 & 690 & 893 & 1150 & 1490 &  & 87 & 736 & 952 & 1230 & 1590 & 2060 &  & 108 & 1010 & 1310 & 1700 & 2200 & 2840 &  & 129 & 1400 & 1810 & 2340 & 3030 & 3920 &  & 150 & 1930 & 2500 & 3230 & 4180 & 5410 &  & 171 & 2670 & 3450 & 4460 & 5770 & 7470 &  & 192 & 3680 & 4760 & 6160 & 7970 & 10300 \\ 
%%
%%&  &  &  &  &  &  &  &  &  &  &  &  &  &  &  &  &  &  &  &  &  &  &  &  &  &  &  &  &  &  &  &  &  &  &  &  &  &  &  &  &  &  &  &  &  &  &  &  &  &  &  &  &  &  &  &  &  &  &  &  & \\
%%
%%\textbf{21} & 100 & 138 & 190 & 263 & 362 &  & 46 & 500 & 690 & 952 & 1310 & 1810 &  & 67 & 690 & 953 & 1310 & 1810 & 2500 &  & 88 & 952 & 1310 & 1810 & 2500 & 3450 &  & 109 & 1310 & 1810 & 2500 & 3450 & 4760 &  & 130 & 1810 & 2500 & 3450 & 4760 & 6570 &  & 151 & 2500 & 3450 & 4760 & 6570 & 9060 &  & 172 & 3450 & 4760 & 6570 & 9060 & 12500 &  & 193 & 4760 & 6570 & 9060 & 12500 & 17200 \\ 
%%\textbf{22} & 138 & 190 & 263 & 362 & 500 &  &  &  &  &  &  &  &  &  &  &  &  &  &  &  &  &  &  &  &  &  &  &  &  &  &  &  &  &  &  &  &  &  &  &  &  &  &  &  &  &  &  &  &  &  &  &  &  &  &  &  &  &  &  &  &  \\ 
%%\textbf{23} & 190 & 263 & 362 & 500 & 690 &  &  &  &  &  &  &  &  &  &  &  &  &  &  &  &  &  &  &  &  &  &  &  &  &  &  &  &  &  &  &  &  &  &  &  &  &  &  &  &  &  &  &  &  &  &  &  &  &  &  &  &  &  &  &  &  \\ 
%%\textbf{24} & 263 & 362 & 500 & 690 & 952 &  &  &  &  &  &  &  &  &  &  &  &  &  &  &  &  &  &  &  &  &  &  &  &  &  &  &  &  &  &  &  &  &  &  &  &  &  &  &  &  &  &  &  &  &  &  &  &  &  &  &  &  &  &  &  &  \\ 
%%\textbf{25} & 362 & 500 & 690 & 952 & 1310 &  &  &  &  &  &  &  &  &  &  &  &  &  &  &  &  &  &  &  &  &  &  &  &  &  &  &  &  &  &  &  &  &  &  &  &  &  &  &  &  &  &  &  &  &  &  &  &  &  &  &  &  &  &  &  & 
%%
%%\end{tabular} 
%%}% Fine scatola 
%%\end{sf}
%%\caption{Misture}
%%
%%\label{defaulttable}
%%\end{table}
%%\end{landscape}
%
%
%
%Il seguente diagramma delle frequenze serve per la comparazione, il numero di ogni mistura  inserito sul punto corrisponde alla frequenza pi bassa.
%
%TABELLA B
%
%Le frequenze sono disegnate lungo la parte alta della partitura. Lo spazio tra le linee corrisponde all'intervallo $\sqrt[25] 5$ da 100 a 17200 \emph{Hz}. Ogni mistura contiene le cinque frequenze a intervallo costante delle quali solo la pi alta e la pi bassa sono indicate da linee orizzontali, unite a inizio e fine da linee verticali. Lo spazio racchiuso tra queste linee viene ombreggiato. Le tre frequenze rimanenti tra la pi alta e la pi bassa vengono trovate dividendo lo spazio racchiuso in 4 intervalli uguali seguendo le linee verticali.  Cos“ alla prima pagina della partitura la prima mistura ha frequenze di 690, 952, 1310, 1810, 2500 (NŒ 67 della tabella delle frequenze), la seconda mistura ha frequenze di 690, 785, 893, 1010, 1150 (NŒ 136 della tabella delle frequenze, a seguire la mistura NŒ 139, 109, 137, 140, etc).
%
%La sovrapposizione di misture di note viene indicata con un'ombreggiatura pi forte a seconda del grado di sovrapposizione.
%
%\section{Volume}
%é stata scelta una scala di intensitˆ di 31 punti con il gradino costante di un decibel tra 0 \emph{dB} e -30 \emph{dB}. La sonoritˆ dell'ascolto a 0 \emph{dB} dipende dalla dimensione della stanza ma non dovrebbe essere inferiore a 80 \emph{phon}. Le cinque parziali di ogni mistura di note hanno la stessa intensitˆ, le misture hanno ognuna curve di inviluppo di crescita e decadimento. C' un limite basso di -40 \emph{dB} per i livelli iniziali o finali delle curve di inviluppo. Il pi alto livello di inviluppo varia tra 0 \emph{dB} e -30 \emph{dB}.\\
%Questi inviluppi sono disegnati in basso alla partitura. Lo spazio tra le linee corrisponde alla variazione di 1 \emph{dB} tra la linea bassa dei -30 \emph{dB} e la linea pi alta a 0 \emph{dB}. La linea pi bassa corrisponde ai -40 \emph{dB}. Ogni mistura nella sezione delle frequenza in alto alla partitura ha un corrispondente inviluppo della stessa lunghezza nella sezione del volume in basso. Lo spazio racchiuso tra la linea di crescita e quella di decadimento di ogni inviluppo e la linea di base (dei -40 \emph{dB})  ombreggiato.
%
%\section{Durata}
%La durata  indicata in \emph{cm} lungo due linee tra le sezioni Altezza e Volume della partitura. Questo valore in centimetri corrisponde alla lunghezza di nastro che scorre a 76.2 \emph{cm} per secondo. Ogni variazione lungo gli assi del tempo  marcata da un segmento verticale tra le due linee. Il numero indicato sotto gli assi si riferisce alla distanza in \emph{cm} tra i segmenti verticali. Quando le misture si sovrappongono, il loro inizio e fine  segnato come una variazione temporale. La durata di una mistura  ottenibile sommando tutte le indicazoni di lunghezza tra il suo inizio e la sua fine (come si pu˜ vedere tra la lunghezza delle linee nella sezione dell'altezza e in quella del volume).
%
%\section{Realizzazione}
%Le cinque frequenze di ogni mistura sono registrate a 0 \emph{db} su nastro (76.2 \emph{cm} per secondo). Di ogni nota registrata su nastro se ne prendono 4 \emph{cm}. I cinque pezzi di nastro da 4 \emph{cm} vengono attaccati uno dietro l'altro su un anello di nastro bianco in ordine dalla pi grave alla pi acuta delle frequenze. Questa sequenza di note viene riprodotta in una stanza con riverberazione di circa 10 secondi con risposta in frequenza regolare e registrata a 0 \emph{db}. Da questo nastro registrato viene tagliata via la sequenza originale delle cinque note (vedi illustrazione).
%
%Il nastro rimanente (vedi Òmistura che usiamoÓ)  tagliato della lunghezza necessaria e messo in loop. Ne viene fatta una copia riproducendolo e contemporaneamente regolando l'inviluppo. Gli inviluppi di crescita vengono ottenuti riproducendo all'indietro il nastro regolato, gli inviluppi di decadimento sono fatti da registrazioni regolate di nastro riprodotte nella direzione originale.
%
%ILLUSTRAZIONE
%
%Il seguente sonogramma mostra la struttura spettrale delle misture a pagina 13 della partitura e come  ottenuto dalla aleatoria modulazione della sequenza sinusoidale dopo la riverberazione.
%
%\onecolumn
%%\chapter{Metodologia Applicata}
%
%La scoperta, la novitˆ che trapela dalla bibliografia, anche se non trova poi conferma nelle note del compositore,  che va eseguita una saturazione analogica (valvolare) in alcuni passaggi su nastro. Non volendo simulare virtualmente una condizione cos“ delicata come la saturazione analogica ho utilizzato un pre-amplificatore valvolare e ne ho poi catturato la risposta all'impulso per poter utilizzarlo con metodi di convoluzione. 
%
%%
%%sadjghasfdhjbgajsdfvnoasdfghadfghasdfoghaosufuighauiosdfghasdfghasdfjkkghadfsjkkghsdfjkghdfsjkghfdj
%%
%
%\section{cSound}
%\subsection{.orc}
%Esempio di file orchestra che simula la generazione dei suoni sinusoidali direttamente montati in mistura .... eccc 
%\begin{quotation}
%\begin{sf}
%
%\indent \textbf{sr}	 = 96000\\
%\indent \textbf{kr}	 = 96000\\
%\indent \textbf{ksmps}  = 1\\
%\indent \textbf{nchnls} = 1\\
%
%\noindent \textbf{instr 1}\\
%\indent ienv1	= (1 / 4) * (1 / 76.2)\\
%\indent iamp1	= 3500
%	
%\indent ienv2	= (1 / 4) * (1 / 76.2)\\
%\indent iamp2	= 15000
%
%\indent ienv3	= (1 / 4) * (1 / 76.2)\\
%\indent iamp3	= 27000
%
%\indent ienv4	= (1 / 4) * (1 / 76.2)\\
%\indent iamp4	= 30000
%
%\indent isus	= 3 * (1 / 76.2)\\
%
%\indent kenv \textbf{linseg} 0, ienv1, iamp1, ienv2, iamp2, ienv3, iamp3, ienv4, iamp4,\\
%\indent \indent \indent isus, iamp4, ienv4, iamp3, ienv3, iamp2, ienv2, iamp1, ienv1, 0\\
%
%\indent a1 \textbf{oscil} kenv, p4, 1\\
%\indent \textbf{out} a1\\
%\textbf{endin}\\
%
%\noindent \textbf{instr 2}\\
%\indent a2 \textbf{oscil} 30000, p4, 1\\
%\indent \textbf{out} a2\\
%\textbf{endin}\\
%
%\end{sf}
%\end{quotation}
%
%\subsection{.sco}
%
%\begin{quotation}
%\begin{sf}
%
%\noindent f1 0 16384 10 1\\
%t0	4572\indent \emph{;	= 76,2 (cm al secondo) * 60 (uguale cicli/beat al minuto)}\\
%
%\indent \emph{;!!! ATTENZIONE !!!	LA MISTURA 39 E LA 49 VANNO IN DISTORSIONE
%\indent ;NON SO PER QUALE MOTIVO...\\
%\indent ;VENGONO RIPRODOTTE ALLA FINE DELLE MISTURE E....\\
%\indent ;FUNZIONANO!\\}
%
%
%
%
%\emph{;i	start		length	freq}
%
%\noindent \emph{;mixture nΠ1\\}
%	\indent i1 	1000		5	 	129\\
%	\indent i1 	1004 	5 		121\\
%	\indent i1 	1008	 	5 		114\\
%	\indent i1 	1012	 	5 		107\\
%	\indent i1 	1016 	5 		100\\
%\emph{;mixture nΠ2\\}
%	\indent i1 	2000		5	 	138\\
%	\indent i1 	2004 	5 		129\\
%	\indent i1 	2008	 	5 		121\\
%	\indent i1 	2012	 	5 		114\\
%	\indent i1 	2016 	5 		107\\
%\emph{;mixture nΠ3}\\
%	\indent i1 	3000		5	 	147\\
%	\indent i1 	3004 	5 		138\\
%	\indent i1 	3008	 	5 		129\\
%	\indent i1 	3012	 	5 		121\\
%	\indent i1 	3016 	5 		114\\
%\emph{;mixture nΠ4}\\
%	\indent i1 	4000		5	 	157\\
%	\indent i1 	4004 	5 		147\\
%	\indent i1 	4008	 	5 		138\\
%	\indent i1 	4012	 	5 		129\\
%	\indent i1 	4016 	5 		121\\
%\emph{;mixture nΠ5\\}
%	\indent i1 	5000		5	 	167\\
%	\indent i1 	5004 	5 		157\\
%	\indent i1 	5008	 	5 		147\\
%	\indent i1 	5012	 	5 		138\\
%	\indent i1 	5016 	5 		129\\
%
%;--------------------------------------------
%
%ecc\ldots\\
%
%\end{sf}
%\end{quotation}
%(vedi file allegato)
%
%%
%%---------------------------------------------------------------------%---------------------------------------------------------------------
%%
%
%\section{La camera di riverberazione}
%
%%
%%---------------------------------------------------------------------%---------------------------------------------------------------------
%%
%
%%\section{Max5}
%%\subsection{il Calcolatore}
%%
%%\begin{figure}[htbp]
%%\begin{center}
%%\includegraphics[width=.8\textwidth]{img/cps2cm_01.png}
%%\caption{Screenshot di Max5 Ð liste}
%%\label{default}
%%\end{center}
%%\end{figure}
%%
%%
%%
%%\begin{figure}[htbp]
%%\begin{center}
%%\includegraphics[width=.8\textwidth]{img/cps2cm_02.png}
%%\caption{Screenshot di Max5 Ð Max window}
%%\label{default}
%%\end{center}
%%\end{figure}
%
%
%%\subsection{Risultati converisione \emph{cps} to \emph{second}}
%%\begin{quotation}
%%\begin{sf}
%%
%%\begin{tabular}{rcl} 
%%page 1\\
%%66.2 cm & = & 0.869 sec\\
%%51.5 cm & = & 0.676 sec\\ 
%%1.3 cm & = & 0.017 sec\\ 
%%57.3 cm & = & 0.752 sec\\ 
%%13.7 cm & = & 0.180 sec\\ 
%%34.6 cm & = & 0.454 sec\\ 
%%58.6 cm & = & 0.769 sec\\ 
%%142.6 cm & = & 1.871 sec\\ 
%%35.0 cm & = & 0.459 sec\\ 
%%30.8 cm & = & 0.404 sec\\ 
%%27.0 cm & = & 0.354 sec\\ 
%%\\
%%page 2\\
%%45.3 cm & = & 0.594 sec\\ 
%%35.0 cm & = & 0.459 sec\\ 
%%198.3 cm & = & 2.602 sec\\ 
%%23.8 cm & = & 0.312 sec\\ 
%%32.8 cm & = & 0.430 sec\\ 
%%16.9 cm & = & 0.222 sec\\ 
%%32.8 cm & = & 0.430 sec\\ 
%%12.4 cm & = & 0.163 sec\\ 
%%39.8 cm & = & 0.522 sec\\ 
%%71.1 cm & = & 0.933 sec\\ 
%%\\
%%page 3\\
%%62.5 cm & = & 0.820 sec\\ 
%%42.5 cm & = & 0.558 sec\\ 
%%41.7 cm & = & 0.547 sec\\ 
%%6.6 cm & = & 0.087 sec\\ 
%%10.6 cm & = & 0.139 sec\\ 
%%15.6 cm & = & 0.205 sec\\ 
%%19.6 cm & = & 0.257 sec\\ 
%%11.7 cm & = & 0.154 sec\\ 
%%25.4 cm & = & 0.333 sec\\ 
%%32.8 cm & = & 0.430 sec\\ 
%%43.1 cm & = & 0.566 sec\\ 
%%98.1 cm & = & 1.287 sec\\ 
%%119.0 cm & = & 1.562 sec\\ 
%%\end{tabular}
%%
%%\begin{tabular}{rcl} 
%%page 4\\
%%111.5 cm & = & 1.463 sec\\ 
%%91.9 cm & = & 1.206 sec\\ 
%%104.6 cm & = & 1.373 sec\\ 
%%32.8 cm & = & 0.430 sec\\ 
%%12.2 cm & = & 0.160 sec\\ 
%%25.4 cm & = & 0.333 sec\\ 
%%7.7 cm & = & 0.101 sec\\ 
%%19.3 cm & = & 0.253 sec\\ 
%%32.8 cm & = & 0.430 sec\\ 
%%100.0 cm & = & 1.312 sec\\ 
%%\\
%%page 5\\
%%55.3 cm & = & 0.726 sec\\ 
%%20.9 cm & = & 0.274 sec\\ 
%%45.3 cm & = & 0.594 sec\\ 
%%30.8 cm & = & 0.404 sec\\ 
%%25.4 cm & = & 0.333 sec\\ 
%%60.7 cm & = & 0.797 sec\\ 
%%66.7 cm & = & 0.875 sec\\ 
%%66.6 cm & = & 0.874 sec\\ 
%%76.5 cm & = & 1.004 sec\\ 
%%9.6 cm & = & 0.126 sec\\ 
%%14.2 cm & = & 0.186 sec\\ 
%%16.6 cm & = & 0.218 sec\\ 
%%16.3 cm & = & 0.214 sec\\ 
%%22.3 cm & = & 0.293 sec\\ 
%%1.2 cm & = & 0.016 sec\\ 
%%\\
%%page 6\\
%%50.8 cm & = & 0.667 sec\\ 
%%46.7 cm & = & 0.613 sec\\ 
%%32.9 cm & = & 0.432 sec\\ 
%%33.8 cm & = & 0.444 sec\\ 
%%52.5 cm & = & 0.689 sec\\ 
%%6.4 cm & = & 0.084 sec\\ 
%%16.2 cm & = & 0.213 sec\\ 
%%27.0 cm & = & 0.354 sec\\ 
%%12.9 cm & = & 0.169 sec\\ 
%%32.4 cm & = & 0.425 sec\\ 
%%19.5 cm & = & 0.256 sec\\ 
%%34.3 cm & = & 0.450 sec\\ 
%%17.2 cm & = & 0.226 sec\\ 
%%15.7 cm & = & 0.206 sec\\ 
%%26.6 cm & = & 0.349 sec\\ 
%%62.6 cm & = & 0.822 sec\\ 
%%35.0 cm & = & 0.459 sec\\ 
%%\end{tabular}
%%
%%\begin{tabular}{rcl} 
%%page 7\\
%%28.9 cm & = & 0.379 sec\\ 
%%55.9 cm & = & 0.734 sec\\ 
%%42.3 cm & = & 0.555 sec\\ 
%%39.8 cm & = & 0.522 sec\\ 
%%108.0 cm & = & 1.417 sec\\ 
%%17.3 cm & = & 0.227 sec\\ 
%%9.1 cm & = & 0.119 sec\\ 
%%2.5 cm & = & 0.033 sec\\ 
%%30.4 cm & = & 0.399 sec\\ 
%%23.8 cm & = & 0.312 sec\\ 
%%32.8 cm & = & 0.430 sec\\
%%12.5 cm & = & 0.164 sec\\ 
%%35.9 cm & = & 0.471 sec\\ 
%%66.7 cm & = & 0.875 sec\\ 
%%\\
%%page 8\\
%%51.5 cm & = & 0.676 sec\\ 
%%75.8 cm & = & 0.995 sec\\ 
%%58.6 cm & = & 0.769 sec\\ 
%%27.6 cm & = & 0.362 sec\\ 
%%27.4 cm & = & 0.360 sec\\ 
%%66.7 cm & = & 0.875 sec\\ 
%%12.2 cm & = & 0.160 sec\\ 
%%18.4 cm & = & 0.241 sec\\ 
%%14.2 cm & = & 0.186 sec\\ 
%%23.8 cm & = & 0.312 sec\\ 
%%12.2 cm & = & 0.160 sec\\ 
%%20.5 cm & = & 0.269 sec\\ 
%%\\
%%page 9\\
%%28.8 cm & = & 0.378 sec\\ 
%%48.3 cm & = & 0.634 sec\\ 
%%22.3 cm & = & 0.293 sec\\ 
%%5.6 cm & = & 0.073 sec\\ 
%%17.3 cm & = & 0.227 sec\\ 
%%9.6 cm & = & 0.126 sec\\ 
%%17.3 cm & = & 0.227 sec\\ 
%%1.5 cm & = & 0.020 sec\\ 
%%1.1 cm & = & 0.014 sec\\ 
%%10.6 cm & = & 0.139 sec\\ 
%%9.7 cm & = & 0.127 sec\\ 
%%14.5 cm & = & 0.190 sec\\ 
%%256.8 cm & = & 3.370 sec\\ 
%%30.8 cm & = & 0.404 sec\\ 
%%25.4 cm & = & 0.333 sec\\ 
%%54.6 cm & = & 0.717 sec\\ 
%%22.3 cm & = & 0.293 sec\\ 
%%32.9 cm & = & 0.432 sec\\ 
%%28.8 cm & = & 0.378 sec\\ 
%%19.6 cm & = & 0.257 sec\\ 
%%25.4 cm & = & 0.333 sec\\ 
%%71.0 cm & = & 0.932 sec\\ 
%%\end{tabular}
%%
%%\begin{tabular}{rcl} 
%%page 10\\
%%37.4 cm & = & 0.491 sec\\ 
%%15.1 cm & = & 0.198 sec\\ 
%%10.3 cm & = & 0.135 sec\\ 
%%3.0 cm & = & 0.039 sec\\ 
%%15.2 cm & = & 0.199 sec\\ 
%%17.3 cm & = & 0.227 sec\\ 
%%70.6 cm & = & 0.927 sec\\ 
%%8.5 cm & = & 0.112 sec\\ 
%%15.3 cm & = & 0.201 sec\\ 
%%45.3 cm & = & 0.594 sec\\ 
%%5.2 cm & = & 0.068 sec\\ 
%%13.9 cm & = & 0.182 sec\\ 
%%13.8 cm & = & 0.181 sec\\ 
%%5.2 cm & = & 0.068 sec\\ 
%%28.9 cm & = & 0.379 sec\\ 
%%27.0 cm & = & 0.354 sec\\ 
%%1.4 cm & = & 0.018 sec\\ 
%%13.8 cm & = & 0.181 sec\\ 
%%9.3 cm & = & 0.122 sec\\ 
%%5.8 cm & = & 0.076 sec\\ 
%%19.6 cm & = & 0.257 sec\\ 
%%11.7 cm & = & 0.154 sec\\ 
%%11.0 cm & = & 0.144 sec\\ 
%%14.4 cm & = & 0.189 sec\\ 
%%2.9 cm & = & 0.038 sec\\ 
%%13.3 cm & = & 0.175 sec\\ 
%%10.5 cm & = & 0.138 sec\\ 
%%32.0 cm & = & 0.420 sec\\ 
%%\\
%%page 11\\
%%35.0 cm & = & 0.459 sec\\ 
%%39.9 cm & = & 0.524 sec\\ 
%%37.3 cm & = & 0.490 sec\\ 
%%45.3 cm & = & 0.594 sec\\ 
%%75.9 cm & = & 0.996 sec\\ 
%%45.3 cm & = & 0.594 sec\\ 
%%35.0 cm & = & 0.459 sec\\ 
%%27.1 cm & = & 0.356 sec\\ 
%%25.3 cm & = & 0.332 sec\\ 
%%5.5 cm & = & 0.072 sec\\ 
%%53.1 cm & = & 0.697 sec\\ 
%%3.9 cm & = & 0.051 sec\\ 
%%17.3 cm & = & 0.227 sec\\ 
%%10.3 cm & = & 0.135 sec\\ 
%%7.2 cm & = & 0.094 sec\\ 
%%\end{tabular}
%%
%%\begin{tabular}{rcl} 
%%page 12\\
%%13.3 cm & = & 0.175 sec\\ 
%%19.6 cm & = & 0.257 sec\\ 
%%11.0 cm & = & 0.144 sec\\ 
%%19.6 cm & = & 0.257 sec\\ 
%%16.2 cm & = & 0.213 sec\\ 
%%65.0 cm & = & 0.853 sec\\ 
%%12.5 cm & = & 0.164 sec\\ 
%%9.0 cm & = & 0.118 sec\\ 
%%13.4 cm & = & 0.176 sec\\ 
%%19.5 cm & = & 0.256 sec\\ 
%%28.8 cm & = & 0.378 sec\\ 
%%13.2 cm & = & 0.173 sec\\ 
%%19.6 cm & = & 0.257 sec\\ 
%%4.6 cm & = & 0.060 sec\\ 
%%17.7 cm & = & 0.232 sec\\ 
%%23.8 cm & = & 0.312 sec\\ 
%%18.4 cm & = & 0.241 sec\\ 
%%116.6 cm & = & 1.530 sec\\ 
%%12.5 cm & = & 0.164 sec\\ 
%%17.2 cm & = & 0.226 sec\\ 
%%6.6 cm & = & 0.087 sec\\ 
%%9.1 cm & = & 0.119 sec\\ 
%%6.5 cm & = & 0.085 sec\\ 
%%25.3 cm & = & 0.332 sec\\ 
%%\end{tabular}
%%
%%\begin{tabular}{rcl} 
%%page 13\\
%%27.1 cm & = & 0.356 sec\\ 
%%16.9 cm & = & 0.222 sec\\ 
%%17.3 cm & = & 0.227 sec\\ 
%%15.6 cm & = & 0.205 sec\\ 
%%2.1 cm & = & 0.028 sec\\ 
%%0.7 cm & = & 0.009 sec\\ 
%%3.9 cm & = & 0.051 sec\\ 
%%19.6 cm & = & 0.257 sec\\ 
%%13.3 cm & = & 0.175 sec\\ 
%%5.1 cm & = & 0.067 sec\\ 
%%8.6 cm & = & 0.113 sec\\ 
%%6.6 cm & = & 0.087 sec\\ 
%%5.7 cm & = & 0.075 sec\\ 
%%22.3 cm & = & 0.293 sec\\ 
%%1.5 cm & = & 0.020 sec\\ 
%%9.1 cm & = & 0.119 sec\\ 
%%0.8 cm & = & 0.010 sec\\ 
%%7.4 cm & = & 0.097 sec\\ 
%%5.1 cm & = & 0.067 sec\\ 
%%4.6 cm & = & 0.060 sec\\ 
%%26.4 cm & = & 0.346 sec\\ 
%%10.3 cm & = & 0.135 sec\\ 
%%15.1 cm & = & 0.198 sec\\ 
%%12.5 cm & = & 0.164 sec\\ 
%%8.4 cm & = & 0.110 sec\\ 
%%8.5 cm & = & 0.112 sec\\ 
%%4.0 cm & = & 0.052 sec\\ 
%%25.4 cm & = & 0.333 sec\\ 
%%19.9 cm & = & 0.261 sec\\ 
%%2.4 cm & = & 0.031 sec\\ 
%%12.7 cm & = & 0.167 sec\\ 
%%5.9 cm & = & 0.077 sec\\ 
%%12.2 cm & = & 0.160 sec\\ 
%%20.7 cm & = & 0.272 sec\\ 
%%7.0 cm & = & 0.092 sec\\ 
%%3.3 cm & = & 0.043 sec\\ 
%%12.5 cm & = & 0.164 sec\\ 
%%1.0 cm & = & 0.013 sec\\ 
%%6.0 cm & = & 0.079 sec\\ 
%%2.0 cm & = & 0.026 sec\\ 
%%11.4 cm & = & 0.150 sec\\ 
%%2.8 cm & = & 0.037 sec\\ 
%%12.7 cm & = & 0.167 sec\\ 
%%18.4 cm & = & 0.241 sec\\ 
%%7.0 cm & = & 0.092 sec\\ 
%%23.8 cm & = & 0.312 sec\\ 
%%7.0 cm & = & 0.092 sec\\ 
%%16.1 cm & = & 0.211 sec\\ 
%%16.7 cm & = & 0.219 sec\\ 
%%14.2 cm & = & 0.186 sec\\ 
%%\end{tabular}
%%
%%\begin{tabular}{rcl} 
%%page 14\\
%%12.9 cm & = & 0.169 sec\\ 
%%11.0 cm & = & 0.144 sec\\ 
%%14.4 cm & = & 0.189 sec\\ 
%%12.0 cm & = & 0.157 sec\\ 
%%4.4 cm & = & 0.058 sec\\ 
%%7.2 cm & = & 0.094 sec\\ 
%%6.8 cm & = & 0.089 sec\\ 
%%4.5 cm & = & 0.059 sec\\ 
%%4.4 cm & = & 0.058 sec\\ 
%%12.9 cm & = & 0.169 sec\\ 
%%7.5 cm & = & 0.098 sec\\ 
%%4.2 cm & = & 0.055 sec\\ 
%%12.5 cm & = & 0.164 sec\\ 
%%2.6 cm & = & 0.034 sec\\ 
%%27.0 cm & = & 0.354 sec\\ 
%%3.8 cm & = & 0.050 sec\\ 
%%25.0 cm & = & 0.328 sec\\ 
%%11.1 cm & = & 0.146 sec\\ 
%%32.9 cm & = & 0.432 sec\\ 
%%2.1 cm & = & 0.028 sec\\ 
%%27.1 cm & = & 0.356 sec\\ 
%%18.2 cm & = & 0.239 sec\\ 
%%11.0 cm & = & 0.144 sec\\ 
%%2.3 cm & = & 0.030 sec\\ 
%%19.6 cm & = & 0.257 sec\\ 
%%4.2 cm & = & 0.055 sec\\ 
%%16.2 cm & = & 0.213 sec\\ 
%%8.0 cm & = & 0.105 sec\\ 
%%6.2 cm & = & 0.081 sec\\ 
%%6.1 cm & = & 0.080 sec\\ 
%%10.1 cm & = & 0.133 sec\\ 
%%13.3 cm & = & 0.175 sec\\ 
%%4.0 cm & = & 0.052 sec\\ 
%%9.3 cm & = & 0.122 sec\\ 
%%1.8 cm & = & 0.024 sec\\ 
%%17.3 cm & = & 0.227 sec\\ 
%%8.5 cm & = & 0.112 sec\\ 
%%16.9 cm & = & 0.222 sec\\ 
%%9.7 cm & = & 0.127 sec\\ 
%%2.0 cm & = & 0.026 sec\\ 
%%6.6 cm & = & 0.087 sec\\ 
%%5.9 cm & = & 0.077 sec\\ 
%%7.5 cm & = & 0.098 sec\\ 
%%7.9 cm & = & 0.104 sec\\ 
%%4.0 cm & = & 0.052 sec\\ 
%%21.0 cm & = & 0.276 sec\\ 
%%10.0 cm & = & 0.131 sec\\ 
%%37.4 cm & = & 0.491 sec\\ 
%%2.5 cm & = & 0.033 sec\\ 
%%\end{tabular}
%%
%%\begin{tabular}{rcl} 
%%page 15\\
%%45.3 cm & = & 0.594 sec\\ 
%%37.3 cm & = & 0.490 sec\\ 
%%2.6 cm & = & 0.034 sec\\ 
%%39.9 cm & = & 0.524 sec\\ 
%%17.3 cm & = & 0.227 sec\\ 
%%5.0 cm & = & 0.066 sec\\ 
%%11.7 cm & = & 0.154 sec\\ 
%%21.2 cm & = & 0.278 sec\\ 
%%15.1 cm & = & 0.198 sec\\ 
%%3.3 cm & = & 0.043 sec\\ 
%%6.6 cm & = & 0.087 sec\\ 
%%3.0 cm & = & 0.039 sec\\ 
%%7.4 cm & = & 0.097 sec\\ 
%%9.8 cm & = & 0.129 sec\\ 
%%1.2 cm & = & 0.016 sec\\ 
%%1.5 cm & = & 0.020 sec\\ 
%%11.0 cm & = & 0.144 sec\\ 
%%7.0 cm & = & 0.092 sec\\ 
%%2.0 cm & = & 0.026 sec\\ 
%%2.3 cm & = & 0.030 sec\\ 
%%10.2 cm & = & 0.134 sec\\ 
%%11.7 cm & = & 0.154 sec\\ 
%%7.9 cm & = & 0.104 sec\\ 
%%7.5 cm & = & 0.098 sec\\ 
%%5.5 cm & = & 0.072 sec\\ 
%%1.2 cm & = & 0.016 sec\\ 
%%3.5 cm & = & 0.046 sec\\ 
%%9.0 cm & = & 0.118 sec\\ 
%%4.7 cm & = & 0.062 sec\\ 
%%3.6 cm & = & 0.047 sec\\ 
%%0.3 cm & = & 0.004 sec\\ 
%%9.6 cm & = & 0.126 sec\\ 
%%2.9 cm & = & 0.038 sec\\ 
%%6.2 cm & = & 0.081 sec\\ 
%%1.9 cm & = & 0.025 sec\\ 
%%6.0 cm & = & 0.079 sec\\ 
%%4.3 cm & = & 0.056 sec\\ 
%%11.7 cm & = & 0.154 sec\\ 
%%12.1 cm & = & 0.159 sec\\ 
%%14.2 cm & = & 0.186 sec\\ 
%%6.7 cm & = & 0.088 sec\\ 
%%9.6 cm & = & 0.126 sec\\ 
%%8.8 cm & = & 0.115 sec\\ 
%%8.0 cm & = & 0.105 sec\\ 
%%6.2 cm & = & 0.081 sec\\ 
%%20.9 cm & = & 0.274 sec\\ 
%%23.8 cm & = & 0.312 sec\\ 
%%11.2 cm & = & 0.147 sec\\ 
%%12.5 cm & = & 0.164 sec\\ 
%%11.3 cm & = & 0.148 sec\\ 
%%\end{tabular}
%%
%%\begin{tabular}{rcl} 
%%page 16\\
%%20.9 cm & = & 0.274 sec\\ 
%%18.4 cm & = & 0.241 sec\\ 
%%23.8 cm & = & 0.312 sec\\ 
%%65.9 cm & = & 0.865 sec\\ 
%%48.3 cm & = & 0.634 sec\\ 
%%58.6 cm & = & 0.769 sec\\ 
%%25.8 cm & = & 0.339 sec\\ 
%%14.1 cm & = & 0.185 sec\\ 
%%18.8 cm & = & 0.247 sec\\ 
%%27.0 cm & = & 0.354 sec\\ 
%%25.3 cm & = & 0.332 sec\\ 
%%31.9 cm & = & 0.419 sec\\ 
%%28.9 cm & = & 0.379 sec\\ 
%%37.3 cm & = & 0.490 sec\\ 
%%22.3 cm & = & 0.293 sec\\ 
%%48.3 cm & = & 0.634 sec\\ 
%%8.2 cm & = & 0.108 sec\\ 
%%\\
%%page 17\\
%%19.6 cm & = & 0.257 sec\\ 
%%11.0 cm & = & 0.144 sec\\ 
%%23.7 cm & = & 0.311 sec\\ 
%%16.3 cm & = & 0.214 sec\\ 
%%32.9 cm & = & 0.432 sec\\ 
%%20.4 cm & = & 0.268 sec\\ 
%%3.4 cm & = & 0.045 sec\\ 
%%41.9 cm & = & 0.550 sec\\ 
%%20.6 cm & = & 0.270 sec\\ 
%%14.4 cm & = & 0.189 sec\\ 
%%27.0 cm & = & 0.354 sec\\ 
%%30.8 cm & = & 0.404 sec\\ 
%%45.0 cm & = & 0.591 sec\\ 
%%80.9 cm & = & 1.062 sec\\ 
%%86.3 cm & = & 1.133 sec\\ 
%%75.8 cm & = & 0.995 sec\\ 
%%\end{tabular}
%%
%%\begin{tabular}{rcl} 
%%page 18\\
%%71.1 cm & = & 0.933 sec\\ 
%%77.0 cm & = & 1.010 sec\\ 
%%3.8 cm & = & 0.050 sec\\ 
%%7.9 cm & = & 0.104 sec\\ 
%%5.6 cm & = & 0.073 sec\\ 
%%64.5 cm & = & 0.846 sec\\ 
%%2.2 cm & = & 0.029 sec\\ 
%%18.7 cm & = & 0.245 sec\\ 
%%32.8 cm & = & 0.430 sec\\ 
%%12.5 cm & = & 0.164 sec\\ 
%%27.3 cm & = & 0.358 sec\\ 
%%3.5 cm & = & 0.046 sec\\ 
%%37.4 cm & = & 0.491 sec\\ 
%%25.4 cm & = & 0.333 sec\\ 
%%43.0 cm & = & 0.564 sec\\ 
%%23.8 cm & = & 0.312 sec\\ 
%%32.9 cm & = & 0.432 sec\\ 
%%9.0 cm & = & 0.118 sec\\ 
%%16.1 cm & = & 0.211 sec\\ 
%%7.7 cm & = & 0.101 sec\\ 
%%\\
%%page 19\\
%%43.5 cm & = & 0.571 sec\\ 
%%11.3 cm & = & 0.148 sec\\ 
%%9.6 cm & = & 0.126 sec\\ 
%%24.4 cm & = & 0.320 sec\\ 
%%8.5 cm & = & 0.112 sec\\ 
%%13.8 cm & = & 0.181 sec\\ 
%%25.4 cm & = & 0.333 sec\\ 
%%19.6 cm & = & 0.257 sec\\ 
%%28.8 cm & = & 0.378 sec\\ 
%%32.9 cm & = & 0.432 sec\\ 
%%146.9 cm & = & 1.928 sec\\ 
%%29.4 cm & = & 0.386 sec\\ 
%%3.4 cm & = & 0.045 sec\\ 
%%9.1 cm & = & 0.119 sec\\ 
%%8.5 cm & = & 0.112 sec\\ 
%%14.2 cm & = & 0.186 sec\\ 
%%14.2 cm & = & 0.186 sec\\ 
%%2.1 cm & = & 0.028 sec\\ 
%%16.3 cm & = & 0.214 sec\\ 
%%21.9 cm & = & 0.287 sec\\ 
%%35.3 cm & = & 0.463 sec\\ 
%%\end{tabular}
%%
%%\begin{tabular}{rcl} 
%%page 20\\
%%35.0 cm & = & 0.459 sec\\ 
%%37.4 cm & = & 0.491 sec\\ 
%%45.3 cm & = & 0.594 sec\\ 
%%56.7 cm & = & 0.744 sec\\ 
%%48.3 cm & = & 0.634 sec\\ 
%%62.5 cm & = & 0.820 sec\\ 
%%55.0 cm & = & 0.722 sec\\ 
%%51.5 cm & = & 0.676 sec\\ 
%%35.2 cm & = & 0.462 sec\\ 
%%27.0 cm & = & 0.354 sec\\ 
%%27.0 cm & = & 0.354 sec\\ 
%%32.9 cm & = & 0.432 sec\\ 
%%\\
%%page 21\\
%%40.9 cm & = & 0.537 sec\\ 
%%6.5 cm & = & 0.085 sec\\ 
%%9.1 cm & = & 0.119 sec\\ 
%%12.5 cm & = & 0.164 sec\\ 
%%23.8 cm & = & 0.312 sec\\ 
%%17.3 cm & = & 0.227 sec\\ 
%%35.8 cm & = & 0.470 sec\\ 
%%16.2 cm & = & 0.213 sec\\ 
%%10.7 cm & = & 0.140 sec\\ 
%%24.3 cm & = & 0.319 sec\\ 
%%8.6 cm & = & 0.113 sec\\ 
%%22.3 cm & = & 0.293 sec\\ 
%%11.5 cm & = & 0.151 sec\\ 
%%36.8 cm & = & 0.483 sec\\ 
%%10.3 cm & = & 0.135 sec\\ 
%%9.3 cm & = & 0.122 sec\\ 
%%7.0 cm & = & 0.092 sec\\ 
%%112.0 cm & = & 1.470 sec\\ 
%%15.2 cm & = & 0.199 sec\\ 
%%15.6 cm & = & 0.205 sec\\ 
%%32.9 cm & = & 0.432 sec\\ 
%%29.9 cm & = & 0.392 sec\\ 
%%5.1 cm & = & 0.067 sec\\ 
%%19.6 cm & = & 0.257 sec\\ 
%%1.2 cm & = & 0.016 sec\\ 
%%\end{tabular}
%%
%%\begin{tabular}{rcl} 
%%page 22\\
%%94.3 cm & = & 1.238 sec\\ 
%%30.8 cm & = & 0.404 sec\\ 
%%19.6 cm & = & 0.257 sec\\ 
%%75.8 cm & = & 0.995 sec\\ 
%%34.2 cm & = & 0.449 sec\\ 
%%6.6 cm & = & 0.087 sec\\ 
%%3.5 cm & = & 0.046 sec\\ 
%%4.0 cm & = & 0.052 sec\\ 
%%18.2 cm & = & 0.239 sec\\ 
%%23.1 cm & = & 0.303 sec\\ 
%%16.2 cm & = & 0.213 sec\\ 
%%62.5 cm & = & 0.820 sec\\ 
%%16.1 cm & = & 0.211 sec\\ 
%%38.3 cm & = & 0.503 sec\\ 
%%6.2 cm & = & 0.081 sec\\ 
%%4.2 cm & = & 0.055 sec\\ 
%%1.3 cm & = & 0.017 sec\\ 
%%18.3 cm & = & 0.240 sec\\ 
%%48.3 cm & = & 0.634 sec\\ 
%%72.5 cm & = & 0.951 sec\\ 
%%\end{tabular}
%%
%%\begin{tabular}{rcl} 
%%page 23\\
%%13.8 cm & = & 0.181 sec\\ 
%%8.5 cm & = & 0.112 sec\\ 
%%10.6 cm & = & 0.139 sec\\ 
%%4.6 cm & = & 0.060 sec\\ 
%%4.4 cm & = & 0.058 sec\\ 
%%22.7 cm & = & 0.298 sec\\ 
%%12.4 cm & = & 0.163 sec\\ 
%%17.3 cm & = & 0.227 sec\\ 
%%5.8 cm & = & 0.076 sec\\ 
%%7.7 cm & = & 0.101 sec\\ 
%%19.4 cm & = & 0.255 sec\\ 
%%16.2 cm & = & 0.213 sec\\ 
%%17.5 cm & = & 0.230 sec\\ 
%%17.3 cm & = & 0.227 sec\\ 
%%10.5 cm & = & 0.138 sec\\ 
%%9.7 cm & = & 0.127 sec\\ 
%%61.2 cm & = & 0.803 sec\\ 
%%4.7 cm & = & 0.062 sec\\ 
%%10.3 cm & = & 0.135 sec\\ 
%%5.7 cm & = & 0.075 sec\\ 
%%16.6 cm & = & 0.218 sec\\ 
%%5.0 cm & = & 0.066 sec\\ 
%%1.7 cm & = & 0.022 sec\\ 
%%14.2 cm & = & 0.186 sec\\ 
%%22.3 cm & = & 0.293 sec\\ 
%%86.3 cm & = & 1.133 sec\\ 
%%55.0 cm & = & 0.722 sec\\ 
%%12.7 cm & = & 0.167 sec\\ 
%%\\
%%page 24\\
%%25.4 cm & = & 0.333 sec\\ 
%%55.0 cm & = & 0.722 sec\\ 
%%59.5 cm & = & 0.781 sec\\ 
%%43.2 cm & = & 0.567 sec\\ 
%%2.1 cm & = & 0.028 sec\\ 
%%3.4 cm & = & 0.045 sec\\ 
%%1.1 cm & = & 0.014 sec\\ 
%%8.0 cm & = & 0.105 sec\\ 
%%17.4 cm & = & 0.228 sec\\ 
%%3.5 cm & = & 0.046 sec\\ 
%%59.0 cm & = & 0.774 sec\\ 
%%18.3 cm & = & 0.240 sec\\ 
%%13.8 cm & = & 0.181 sec\\ 
%%105.2 cm & = & 1.381 sec\\ 
%%42.5 cm & = & 0.558 sec\\ 
%%12.5 cm & = & 0.164 sec\\ 
%%28.8 cm & = & 0.378 sec\\ 
%%9.6 cm & = & 0.126 sec\\ 
%%\end{tabular}
%%
%%\begin{tabular}{rcl} 
%%page 25\\
%%45.3 cm & = & 0.594 sec\\ 
%%8.0 cm & = & 0.105 sec\\ 
%%14.2 cm & = & 0.186 sec\\ 
%%4.5 cm & = & 0.059 sec\\ 
%%28.6 cm & = & 0.375 sec\\ 
%%71.1 cm & = & 0.933 sec\\ 
%%12.8 cm & = & 0.168 sec\\ 
%%12.5 cm & = & 0.164 sec\\ 
%%8.4 cm & = & 0.110 sec\\ 
%%13.9 cm & = & 0.182 sec\\ 
%%7.9 cm & = & 0.104 sec\\ 
%%37.4 cm & = & 0.491 sec\\ 
%%26.1 cm & = & 0.343 sec\\ 
%%25.4 cm & = & 0.333 sec\\ 
%%9.8 cm & = & 0.129 sec\\ 
%%1.2 cm & = & 0.016 sec\\ 
%%16.1 cm & = & 0.211 sec\\ 
%%2.5 cm & = & 0.033 sec\\ 
%%4.7 cm & = & 0.062 sec\\ 
%%32.9 cm & = & 0.432 sec\\ 
%%21.2 cm & = & 0.278 sec\\ 
%%42.3 cm & = & 0.555 sec\\ 
%%3.0 cm & = & 0.039 sec\\ 
%%15.4 cm & = & 0.202 sec\\ 
%%3.0 cm & = & 0.039 sec\\ 
%%39.5 cm & = & 0.518 sec\\ 
%%\\
%%page 26\\
%%27.1 cm & = & 0.356 sec\\ 
%%44.0 cm & = & 0.577 sec\\ 
%%55.0 cm & = & 0.722 sec\\ 
%%64.0 cm & = & 0.840 sec\\ 
%%62.5 cm & = & 0.820 sec\\ 
%%32.8 cm & = & 0.430 sec\\ 
%%119.0 cm & = & 1.562 sec\\ 
%%45.3 cm & = & 0.594 sec\\ 
%%86.3 cm & = & 1.133 sec\\ 
%%\end{tabular}
%%
%%
%%
%%\end{sf}
%%\end{quotation}
%%
%%\section{Editing \& Matering}
%%
%%\chapter{Comparazione \& Conclusioni}
%%
%%\chapter{Diagramma a blocchi}
