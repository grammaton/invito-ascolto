%!TEX TS-program = xelatex
%!TEX encoding = UTF-8 Unicode
% !TEX root = ../../2017-GS-COME01-INVITO-ASCOLTO.tex

\clearpage

\thispagestyle{empty}

\includepdf[scale=1.03,
		    pagecommand={
		    	\begin{tikzpicture}[
					remember picture,
					overlay]
		    	\node [xshift=2cm,yshift=1cm] at (current page.south west) {\color{white}{\emph{Heinz \textbf{Karnine}}}};
				\end{tikzpicture}}
		    ]{images/stockhausen/stockhausen.pdf}

\clearpage

%-------------------------------------------------------------
%------------------------- KARLHEINZ STOCKHAUSEN - STUDIE II -
%-------------------------------------------------------------

\chapter*{1956. Karlheinz Stockhausen. \\ \emph{Studie II}.}
\addcontentsline{toc}{chapter}{1956. Karlheinz Stockhausen. \emph{Studie II}.}

	\begin{flushright}
		\textit{Nella nostra anima c'\`e una incrinatura che, se sfiorata, \\
		risuona come un vaso prezioso riemerso dalle profondit\`a della terra} \\
		Wassilly Kandinsky - \emph{Lo Spirituale nell'Arte}
	\end{flushright}

	\begin{flushright}
		\textit{Music of Changes // John ChAnGEs} \\
		Pierre Boulez
	\end{flushright}

	\begin{flushright}
		\textit{Si dice che i compositori abbiano orecchio per la musica e \\
		di solito significa che non sentono nulla che arrivi alle loro orecchie. \\
		Le loro orecchie sono murate dai suoni di loro creazione.} \\
		John Cage - \emph{45' for a Speaker} (1954)
	\end{flushright}

\bigskip

\begin{multicols}{2}

	La prima volta che ho realizzato lo \emph{Studie II} di Stockhausen è stato nel 2008 per l'esame di \emph{Analisi e Sintesi I}
	al mio primo anno del \emph{Triennio di Musica Elettronica} a Roma. Ho recuperato i miei appunti di realizzazione, rielaborato la procedura che
	ripropongo in questo capitolo con qualche accorgimento in più.

\begin{quote}
	Ogni opera d'arte \`e figlia del suo tempo, e spesso \`e madre dei nostri
	sentimenti.

	Analogamente, ogni periodo culturale esprime una sua arte, che non si ripeter\`a mai pi\`u.
	Lo stesso sforzo di ridar vita a principi estetici del passato pu\`o creare al massimo delle opere d'arte che sembrano bambini nati morti.
	Noi non possiamo, ad esempio, avere la sensibilit\`a e la vita interiore\footnote{Wassilly Kandinsky, \emph{Lo Spirituale nell'Arte}, SE. 1989}\ldots
\end{quote}

%[\ldots] Il suono sinusoidale  un fenomeno completamente nuovo nella musica. Si intende un suono senza armonici, cio il suono nudo, che forma una vibrazione sinusoidale. é un prodotto dei vibratori graduati elettronici, con i quali di solito vengono prodotti nelle stazioni radio il segnale orario, il diapason per i musicisti e Ð prima e dopo l'orario giornaliero di trasmissione Ð le frequenze pilota. [\ldots] é veramente senza luogo, una vibrazione isolata, che nasce ed  amplificata nella valvola elettronica e diventa udibile solo nell'altoparlante. Il suono sinusoidale si sottrae a  ogni definizione all'infuori di quella con il numero di \emph{Hertz} delle sue vibrazioni. Girando l'interruttore del condensatore del generatore elettronico lo si pu˜ spostare in qualsiasi punto dell'intera sfera uditiva. Il suo suono, secondo la precisa descrizione di Herbert Eimert, 
%\begin{quotation}
%\begin{sf}
%\begin{small}
%\noindent neutrale e giˆ abbastanza forte a un'intensitˆ media, perci˜ non  affatto smorto e insussistente. Siccome manca degli armonici caratteristici, non ha alcun timbro ben definito. Il suo contrassegno principale  la diretta immediatezza del suono. Dipende dalla sua natura elettrica il fatto che esso risuona come una corrente uniforme, rigido e non modulato.
%\end{small}
%\end{sf}
%\end{quotation}
%
%\noindent [\ldots] Suoni sinusoidali possono essere sovrapposti in qualsiasi numero e frequenza, sia armonicamente, di modo che si ottiene un suono con un corteggio di armonici naturale, sia non armonicamente, di modo che nasce una cosiddetta <<~mescolanza di suoni~>>. [\ldots] L'inizio del suono coincide con l'innesto, e con il disinnesto s'interrompe immediatamente. Perci˜ il carattere del suono deve essere aggiunto alla composizione e per fare questo si offrono le pi varie possibilitˆ. Con le forbici il compositore taglia via dal nastro inciso non solo la lunghezza ma anche le cosiddette <<~curve di inviluppo~>>, le caratteristiche forme dei suoi elementi acustici. [\ldots] Inoltre egli pu˜ servirsi di un apparecchio di risonanza o della riverberazione che si trova in ogni stazione radio. Se la formazione lo soddisfa, inizia allora il montaggio dei molti piccoli ritagli di nastro nella disposizione ritmica desiderata.
%
%Complicatissimi passaggi dinamici nascono sotto un costante controllo con il centimetro. Siccome il compositore conosce la velocitˆ del nastro, di solito 76,2 o 38,1 centimetri al secondo, misurando la lunghezza di pause e suoni egli pu˜ produrre ritmi <<~irrazionali~>> precisissimi, mentre ogni compositore di musica strumentale fallirebbe senza speranza in quest'impresa.
%
%[\ldots] L'uomo irruppe nel regno della macchina, nella sfera a essa peculiare della trasformazione dell'energia, che in un primo tempo serviva soltanto a facilitare e accelerare il suo lavoro.
%
%[\ldots] Esistono diverse possibilitˆ di fissare la musica elettronica in qualcosa di simile a uno <<~spartito~>>. Una <<~partitura~>> elettronica, lo \emph{Studio II} di Karlheinz Stockhausen,  giˆ stampata. Un progetto di costruzione, un disegno tecnico per cos’ dire, documento unico di una musica dell'avvenire. Al primo sguardo si presenta come un disegno affascinante di vari rettangoli e triangoli.
%
%L'idea della notazione, cio la distribuzione del tempo in senso orizzontale da sinistra a  destra, e la collocazione delle note e dei suoni in senso verticale, dal basso in alto, rimane inalterata. Invece  nuova l'annotazione assoluta. Le note musicali si limitano a dare un sistema di riferimento. Esse sono relative. Il loro valore assoluto dipende da una convenzione non obbligatoria, dall'altezza relativa del diapason. Un'esecuzione pi o meno autentica Ð anche per quel che riguarda l'altezza di suono Ð riesce unicamente se si ha una precisa conoscenza di questa convenzione. Invece la musica elettronica  incisa su nastro conformemente alle idee del compositore. L'interpretazione non  nŽ necessaria nŽ possibile. Perci˜ la rappresentazione grafica dell'opera elettronica deve essere vera e assolutamente precisa: il tecnico dello studio lavora su  di essa. Molte novitˆ saltano agli occhi. L'altezza del suono si misura in \emph{Hertz}, cio in vibrazioni al secondo; l'intensitˆ si esprime in \emph{Decibel}, cio nei gradi di un aumento del 26 per cento circa dell'energia sonora, che vengono ancora percepiti dall'orecchio Ð non pi con indicazioni cos“ vaghe come forte e pianissimo; per la velocitˆ si evitano le arbitrarie indicazioni andante, presto o largo, si calcola secondo i centimetri del nastro che scorre. Al posto dell'approssimazione interpretativa  subentrata matematica esattezza. La <<~partitura~>> elettronica di Stockhausen corrisponde alle premesse di un caso compositivo speciale: le centonovantatre mescolanze di suoni, di cinque suoni sinusoidali ciascuno, che egli usa, non sono montati singolarmente ma risultano dalla relazione meccanica dei suoni sinusoidali sonati uno dopo l'altro nella riverberazione. Siccome si tratta esclusivamente di  suoni di uguale intensitˆ e con intervalli costanti, ne consegue subito che una simile annotazione semplificata pu˜ valere soltanto  per questa composizione elettronica.
%
%Se si esamina ora nei particolari, si vede che la parte superiore della <<~partitura~>> Ð un po' pi della metˆ Ð  costituita da un sistema di linee che comprende lo spazio da 100 fino a 17200 \emph{Hertz} sfruttato da Stockhausen nello \textit{Studio II}. Rettangoli di varie forme simbolizzano blocchi di cinque suoni sinusoidali ciascuno, cio mescolanze di suoni. Sotto questa parte la lunghezza dei blocchi  riportata ancora una volta su di una scala in centimetri di nastro. 76,2 centimetri corrispondono alla velocitˆ allora usuale del nastro. Pi sotto ancora vi  un secondo pi sottile sistema di linee per la dinamica da 40\footnote{in realtˆ  da Ð40 a 0 \emph{dB}} fino a 0 decibel. L'altezza del grafico in questo sistema indica l'intensitˆ corrispondente. Le sue forme determinano i contorni, cio le curve di inviluppo delle mescolanze dei suoni. In alcuni punti dei due sistemi mancano figure, il che significa pausa. Anche la durata  visibile dal numero nella scala delle lunghezze. Una pagina della partitura rappresenta all'incirca sei secondi di musica.
%
%I primi sette pezzi elettronici dello studio di Colonia [\ldots] Sono come uno scenario acustico, che provoca senza dubbio eccitamenti molto violenti i quali, pur non causando uno shock, opprimono tuttavia in modo quasi insopportabile il sistema nervoso.
%
%\section*{da \textit{La Musica Elettronica\footnote{La Musica Elettronica Ð Pousseur  (pag. 65 - 68)}}}
%[\ldots] L'articolazione e l'esatto metodo di produzione del materiale sono stati interamente esposti nella prefazione alla ÒpartituraÓ, una delle rare rappresentazioni di musica elettronica pubblicate: si potrˆ rimandare ad essa. In effetti, in essa
%si ottiene una fusione molto maggiore degli ÒelementiÓ all'interno dei Òsuoni complessiÓ. Essa  dovuta, in primo luogo, alla loro Òuguaglianza dinamicaÓ ed alla complessitˆ molto pi grande dei loro rapporti armonici (ricordiamo che nel \textit{Primo Studio} non si trattava di spettri ÒarmoniciÓ propriamente detti, centrati su un fondamentale unico; almeno tutti gli intervalli facenti parte di un blocco erano degli intervalli giusti, degli intervalli semplici e ÒtrasparentiÓ). Essa 
%anche dovuta, per alcuni di essi (per i pi ÒagglomerantiÓ), al serrarsi di questi intervalli costitutivi che provoca (non soltanto per la percezione, ma, si potrebbe dire, oggettivamente) la distruzione reciproca delle periodicitˆ individuali e (come giˆ il cromatismo simultaneo nelle musiche strumentali di cui abbiamo parlato nel primo capitolo) genera dei veri rumori (molto controllati, tuttavia). Inoltre, l'utilizzazione della camera a eco (ÒnaturaleÓ) come uno degli stessi mezzi della produzione sonora (metodo che reintroduce fin da allora un elemento non elettronico e per questo non strettamente controllato) rinforza ancora la fusione delle componenti (valorizzando i legami interferenziali) e conferisce ai risultati un'unitˆ supplementare, dovuta al colore proprio di cui riveste in un certo senso tutto ci˜ che
%passa attraverso di essa.
%
%Infine, l'attacco simultaneo di diversi blocchi di questa specie da cui si ritagliano eventualmente le regioni armoniche (e la cui ulteriore evoluzione divergente dˆ, per contrasto, conferma), ed il fatto che Stockhausen abbia superato su questi attacchi di un poco le soglie di registrazione del nastro magnetico (di passare nella zona ÒrossaÓ dei potenziometri: fino a + 6dB), producendo, con la distorsione che ne deriva, dei veri transitori, paragonabili agli attacchi di certi strumenti (a percussione), contribuiva ad ottenere dei fenomeni sonori molto pi unitari, la cui unitˆ era giˆ caratterizzata da un certo tasso di evoluzione interna, e li rendeva capaci di sopportare meglio il paragone con il carattere ÒorganicoÓ dei fenomeni naturali. Certo, a questo si univa una (molto relativa, ma innegabile nel rigore della prospettiva iniziale) perdita di controllo. Tuttavia le immediate conclusioni che se ne sarebbero potute tirare, le conseguenze che questo
%avrebbe avuto, nel senso di un certo ammorbidimento dei principi di realizzazione (e in primo luogo di concezione) non erano il solo insegnamento che se ne potesse trarre: Stockhausen (e coloro che seguivano da vicino le sue esperienze cominciando eventualmente a dedicarsi ad esperienze parallele) aveva appreso qui ogni sorta di nozioni precise sulla struttura dei fatti sonori, sulla possibilitˆ di continuare a ricercarne il controllo integrale, anche se questo dovesse durare per un periodo abbastanza lungo e passare attraverso tappe apparentemente in contraddizione.
%
%In effetti, fin dal compimento del \textit{Primo Studio} di Stockhausen, altri compositori erano stati invitati a lavorare temporeneemente allo studio di Colonia ed a realizzarvi una composizione per forza di cose modesta. Cosi, alla fine del 1954, una prima esecuzione dei lavori (che occupavano la metˆ di un concerto) potŽ venir proposta al pubblico (l'altra parte comprendeva esecuzioni di nuova musica americana da parte di John Cage e di David Tudor). Ad eccezione di qualche dettaglio, tuttavia, nessuno di questi lavori apportava nulla di fondamentalmente nuovo rispetto alle realizzazioni di Stockhausen.
%
%[\ldots] fenomeni paragonabili fino a un certo punto agli aggregati pi indivisibili del \textit{Secondo Studio} di Stockhausen, potevano essere ottenuti filtrando un fenomeno elettronico il meno periodico, il pi disordinato possibile, la cui applicazione acustica si chiama Òrumore biancoÓ. Ed infine alcuni momenti dell'una o dell'altra composizione particolarmente movimentati, particolarmente ÒmicrostrutturatiÓ provavano (soprattutto se accelerati ancora, cosa di cui si poteva fare esperienza quotidiana durante il lavoro di studio) che si potevano raggiungere unitˆ di una nuova specie, la cui instabilitˆ, la cui mobilitˆ, sarebbe una delle caratteristiche principali, che le contrapporrebbe dunque in maniera molto netta alla maggior parte dei suoni strumentali (analoghe solo alle misture a un tempo molto dense e molto rapide, come ne esistevano giˆ, per esempio nella seconda delle prime \textit{Structures} per due pianoforti di Boulez, o nei \textit{Kontrapunkte} di Stockhausen). Prescindendo dalla perdita di controllo che il tentativo di sistematizzare dei fenomeni di questo tipo poteva rappresentare (e che potrebbe, per lo meno provvisoriamente, essere compensato da criteri di determinazione statistica) era proprio questo l'effetto di una di quelle Òimmaginazioni concreteÓ di cui dicevamo, che furono all'origine della nascita della musica seriale generalizzata, immaginazione che ci sforziamo dunque, intravedendola, di rendere attuale con un massimo di efficacia, ad un tempo ÒespressivaÓ (cio piuttosto ÒqualitativaÓ) e strutturale (o pi ÒquantitativaÓ). Sembrava che ci fossero delle vie per condurre a questo senza passare per il missaggio ed il montaggio di elementi sinusoidali (operazione naturalmente fastidiosa nel caso in cui si vogliano realizzare dei fenomeni formalmente antinomici rispetto alla sinusoide Ð e del resto poco efficaci a causa dei Òrumori di fondo,Ó nel senso molto generale della parola, che la molteplicitˆ delle operazioni introduce ed accumula). Le sinusoidi non rappresentarono dunque pi che uno degli estremi di un campo di cui gli altri due ÒpoliÓ simbolici sembrano essere: l'onda periodica ÒangolareÓ cio il meno sinusoidale possibile, definibile in termini di ÒimpulsoÓ (dente di sega, ÒagoÓ o rettangolo) ed il Òrumore biancoÓ o processo vibratorio il meno periodico possibile.
%
%Questo allargamento delte ÒriserveÓ materiali a cui poteva attingere il compositore apriva ad un tratto alla musica elettronica uno spazio figurativo molto pid ricco, molto pi duttile. Se i compositori sapessero rivelarsi sufficientemente attenti, la luce momentanea (ed in alcuni casi del tutto relativa) del principio del controllo integrale, potrebbe non essere che una specie di astuzia per permettere l'appropriazione e la progressiva realizzazione di questi materiali sonori in diverse tappe. Tuttavia la prima di esse sarebbe una tappa in cui l'accento sarebbe posto pi sovente, perlomeno per una parte dei compositori (e non necessariamente per i meno ÒprospetticiÓ), sull'aspetto qualitativo, spontaneo, e quindi sulla generazione relativamente empirica, talvolta davvero improvvisata, delle nuove sonoritˆ).

\end{multicols}
