%!TEX TS-program = xelatex
%!TEX encoding = UTF-8 Unicode
% !TEX root = ../../2017-GS-COME01-INVITO-ASCOLTO.tex

\clearpage

\thispagestyle{empty}

\includepdf[scale=1.03,
		    pagecommand={
		    	\begin{tikzpicture}[
					remember picture,
					overlay]
		    	\node [xshift=2cm,yshift=1cm] at (current page.south west) {\color{white}{\emph{Heinz \textbf{Karnine}}}};
				\end{tikzpicture}}
		    ]{images/stockhausen/stockhausen.pdf}

\clearpage

%-------------------------------------------------------------
%---------------------------- KARLHEINZ STOCKHAUSEN - MANTRA -
%-------------------------------------------------------------

\chapter*{1970. Karlheinz Stockhausen.\\\emph{Mantra}.}
\addcontentsline{toc}{chapter}{1970. Karlheinz Stockhausen. \emph{Mantra}.}

%	\begin{flushright}
%		\textit{Nella nostra anima c'è una incrinatura che, se sfiorata, \\
%		risuona come un vaso prezioso riemerso dalle profondità della terra} \\
%		Wassilly Kandinsky - \emph{Lo Spirituale nell'Arte}
%	\end{flushright}
%
%	\begin{flushright}
%		\textit{Music of Changes // John ChAnGEs} \\
%		Pierre Boulez
%	\end{flushright}
%
%	\begin{flushright}
%		\textit{Si dice che i compositori abbiano orecchio per la musica e \\
%		di solito significa che non sentono nulla che arrivi alle loro orecchie. \\
%		Le loro orecchie sono murate dai suoni di loro creazione.} \\
%		John Cage - \emph{45' for a Speaker} (1954)
%	\end{flushright}

\bigskip

%\begin{multicols}{2}

In una presentazione del 2009 per la prima neozelandese di \emph{Mantra}, Robin Maconie traccia i collegamenti storici tra la genialità di Stockhausen e una serie di semine musicali che negli anni precedenti ne hanno costruito il background immaginifico.

Il primo nome che Maconie collega a Stockhausen è quello di Arnold Schoenberg, mettendo in relazione il gli intervalli iniziali di \emph{Mantra} con alcuni tratti melodici dell'inizio dell'\emph{Op. 11 N. 2} per pianoforte. Maconie mette in relazione la nota perno \emph{La} attorno alla quale si muovono formule simmetriche con l'\emph{Op. 27} di Webern e \emph{Music for String, Percussion e Celesta} di Bartok:

\begin{quote} The mirror-symmetry in both cases is significant, and also the date: both works were composed in 1936. For the postwar Darmstadt generation of serial composers, mirror-imagery takes the form of a symmetrical all-interval series, the utopian generative principle commended by Herbert Eimert, Stockhausen's mentor and coeditor of the periodical Die Reihe, and adopted by Boulez, Stockhausen, and Nono. [\ldots] In Mantra a listener is able to detect echoes and allusions to a western tradition of music-making.\footnote{La simmetria speculare in entrambi i casi è significativa, e lo è anche la data: entrambe le opere furono composte nel 1936. Per la generazione postbellica dei compositori serialisti di Darmstadt, le immagini speculari presero la forma di una serie simmetrica che contemplasse tutti gli intervalli, un principio generativo utopico proposto da Herbert Eimert, mentore e coeditore di Stockhausen del periodico \emph{Die Reihe}, e adottato da Boulez, Nono e Stockhausen stesso. [\ldots] In \emph{Mantra} l'ascoltatore è in grado di riconoscere gli echi e le allusioni alla tradizione musicale occidentale.}
\end{quote}

Il collegamento tra Stockhausen e Boulez passa anche per la costruzione formale, per gli sviluppi interni della struttura che si contrae ed espande per tutta l'estensione della tastiera all'interno delle tessiture musicali, che Maconie collega alle \emph{Structures} per due pianoforti del 1951-52.

L'ultima riflessione di Maconie sulle radici storiche di \emph{Mantra} è relativa alla presenza delle percussioni. I nomi che qui collega sono quelli di John Cage, che negli anni cinquanta raggiunge l'Europa con le sue sonorità pianistiche non convenzionali, e di Bela Bartok con la \emph{Sonata per due pianoforti e percussione} sulla quale nel 1951 Stockhausen redige la sua tesi di diploma musicale.


%1969 quattro persone che chiacchierano di questo e quello in una macchina tra medicine connecticut e boston.
%lungo la strada scrive su una busta che aveva in tasca una melodia che contiene tutte e dodici le note. un anno dopo
%inizia il suo lavoro per due pianoforti e riprende questa melodia.
%tutta la melodia stirata sull'intera durata del brano, un'ora. e contemporanemamente compressa nella più piccola portzione temporale.
%ogni nota a sua volta richiamava a se tutte le altre note rendendo per ognuno di questi punti il complesso delle 12 note. Il tutto somiglia
%molto ad un sistema di stelle.
%
%tutta la melodia è il mantra, come fosse una formula. ci sono 4 regioni separate da pause. la prima regione è formata da 4 note. la seconda da 2. la terza da 5 la quarta da 3
%
%mirror
%
%spiegherò come il mantra, la formula, può essere usata per l'intera composizione, per fare questo ho bisogno della variazione
%
%trovare qualcosa sulle lezioni inglesi?
%
%dobbiamo immaginare come se ogni nota determina un'intera sezione di una certa composizione
%
%combinare le sezioni tra loro con la tecnica delle variazioni
%
%non ho costruito la formula sulla base di una scala cromatica

%due note su mantra:
%
%\emph{perché le onde corte?} \\
%Stockhausen inizia a scrivere \emph{Mantra} durante la residenza giapponese di Osaka per il settantesimo Expo universale. Utilizza le onde corte nel 1968 per \emph{Kurzwellen} e \emph{Spiral} e proprio questo brano è cardine per la programmazione dei concerti Expo. La prima di \emph{Mantra} avviene al \emph{Donaueschingen Festival} con il duo Kontarksy nel 1970, dopo l'esperienza di Osaka. \emph{Mantra} è il primo brano in cui Stockhausen utilizza la \emph{formula}. L'idea di applicare la \emph{Ring Modulation} sui suoni di pianoforte proviene quindi dai precedenti lavori: \emph{Kurzwellen mit Beethoven} dove tra i vari suoni di Beethoven modula anche le sonate per pianoforte. Anche le onde corte provengono dai lavori di quel periodo, \emph{Kurzwellen} e \emph{Spiral}.

%\end{multicols}

%http://www.sonoloco.com/rev/stockhausen/16.html
%http://stockhausenspace.blogspot.it/2014/06/opus-32-mantra.html
%http://stockhausenspace.blogspot.it/p/timeline-history-of-20th-century.html
%http://stockhausenspace.blogspot.it/p/year-biographical-info-from-official.html
%https://www.youtube.com/watch?v=Z8srbuxyIVw&feature=youtu.be
%http://www.staff.city.ac.uk/newton.armstrong.1/mantra/
%https://www.swr.de/swr-classic/donaueschinger-musiktage/programme/1970/-/id=2136962/did=3459926/nid=2136962/1b1tc0l/index.html

\begin{table}[htp]
\begin{center}
\begin{sf}{\footnotesize
\begin{tabular}{r c c c c c c c c c c c c c c c c c c }

       \textbf{altezze} & \multicolumn{4}{c}{4} 	 & \multirow{3}*{pausa 3} & \multicolumn{3}{c}{2} & \multirow{3}*{pausa 2} & \multicolumn{5}{c}{5}  & \multirow{3}*{pausa 1} & \multicolumn{3}{c}{3}  \\
\textbf{durata regione} & \multicolumn{4}{c}{10} &                        & \multicolumn{3}{c}{6} &                        & \multicolumn{5}{c}{15} &                        & \multicolumn{3}{c}{12}   \\
  \textbf{suddivisione} & 1 & 2 & 3 & 4          &                        & 2 & 2 & 2             &                        & 5 & 2 & 1 & 3 & 4      &                        & 4 & 2 & 6 \\

\end{tabular}}
\end{sf}
\end{center}
\caption{Melodia - Formula - Regioni}
\label{default}
\end{table}%


%----------------------

\begin{table}[htp]
\begin{center}
\begin{sf}{\footnotesize
\begin{tabular}{|c|l|c|c|}

\hline
& Andamento musicale & Battuta iniziale & Nota di partenza \\
\hline
\hline
1 & Ripetizione regolare & 12 & La \\
\hline
2 & Accento alla fine & 61 & Si \\
\hline
3 & \emph{Normale} & 89 & Sol\# \\
\hline
4 & gruppetti e acciaccature attorno ad un tono centrale & 125 & Mi \\
\hline
5 & Tremolo & 205 & Fa \\
\hline
6 & Accordo (accentato) & 282 & Re \\
\hline
7 & Accento all'inizio & 445 & Sol\# \\
\hline
8 & Connessione cromatica & 488 & Mi\emph{b} \\
\hline
9 & Staccato & 538 & Re\emph{b} \\
\hline
10 & Ripetizioni irregolari (Codice \emph{Morse}) & 587 & Do \\
\hline
11 & Trilli & 611 & Si\emph{b} \\
\hline
12 & Accento \emph{Sforzando} & 641 & Sol\emph{b} \\
\hline
13 & Arpeggio come connessione & 673 & La \\
\hline

\end{tabular}}
\end{sf}
\end{center}
\caption{\emph{Mantra} consta di 13 grandi sezioni derivanti dalle 13 note della formula. Ogni sezione tratta un particolare andamento musicale, anch'esso derivato dal mantra.}

\label{default}
\end{table}%


%----------------------

\begin{table}[htp]
\begin{center}
\begin{sf}{\footnotesize
\begin{tabular}{|c|c|c|c|c|c|c|c|c|c|c|c|c|c|}

\hline
  & \textbf{I} & \textbf{II} & \textbf{III} & \textbf{IV} & \textbf{V} & \textbf{VI} & \textbf{VII} & \textbf{VIII} & \textbf{IX} & \textbf{X} & \textbf{XI} & \textbf{XII} & \textbf{XIII} \\
  \hline
  \hline
\textbf{1} & 12 & 61 & 89 & 125 & 205 & 282 & 446 & 488 & 540 & 587 & 611 & 641 & 673 \\
\hline
\textbf{2} & 56 & 85 & 122 & 198 & 276 & 318 & 486 & 535 & 538 & 609 & 630 & 671 & 871 \\
\hline
\textbf{3} & 12 & 61 & 102 & 125 & 209 & 286 & 447 & 492 & 582 & 590 & 612 & 661 & 676 \\
\hline
\textbf{4} &  53 & 83 & 121 & 195 & 268 & 318 & 478 & 530 & 570 & 608 & 628 & 668 & 868 \\
\hline
\textbf{5} & 46 & 81 & 114 & 192 & 262 & 312 & 450 & 526 & 567 & 594 & 626 & 665 & 865 \\
\hline
\textbf{6} & 26 & 75 & 91 & 146 & 257 & 311 & 454 & 525 & 563 & 603 & 620 & 634 & 855 \\
\hline
\textbf{7} & 29 & 79 & 110 & 189 & 262 & 300 & 452 & 508 & 556 & 607 & 624 & 665 & 862 \\
\hline
\textbf{8} & 19 & 71 & 97 & 128 & 245 & 296 & 472 & 496 & 544 & 599 & 616 & 656 & 681 \\
\hline
\textbf{9} & 22 & 73 & 100 & 201 & 280 & 304 & 466 & 502 & 540 & 609 & 618 & 659 & 685 \\
\hline
\textbf{10} & 59 & 87 & 124 & 132 & 246 & 325 & 458 & 537 & 586 & 610 & 632 & 672 & 872 \\
\hline
\textbf{11} & 30 & 77 & 105 & 152 & 259 & 308 & 486 & 521 & 548 & 603 & 622 & 663 & 858 \\
\hline
\textbf{12} & 13 & 69 & 92 & 127 & 212 & 292 & 448 & 493 & 562 & 592 & 614 & 651 & 679 \\
\hline

\end{tabular}}
\end{sf}
\end{center}
\caption{Tabella dei 156 \emph{mantra} (battute di inizio). La riga con le cifre romane indica le tredici sezioni mentre la prima colonna numera le dodici suddivisioni in sezioni. Da questa visualizzazione si possono notare alcune particolari condizioni come per esempio la partenza delle espansioni I.1 e I.3 alla stessa battuta. (da Richard Toop, \emph{Six Lectures from the Stockhausen Courses Kurten 2002})}

\label{default}
\end{table}%

\begin{lstlisting}[style=SuperCollider-IDE]
SynthDef('shortwave', {
	var level, sw1, sw1_d, sw1_f, sw1_t, sw2, sw2_d, sw2_t;

	level = Lag3.kr(In.kr(2).linlin(0, 1, -30, 0), 0.1).dbamp;

	sw1_d = Drand(({ rrand(0.02, 0.04) } ! 23), inf);
	sw1_t = Duty.kr(sw1_d, 0, Dseq([1, 0], inf));
	sw1_f = TChoose.kr(sw1_t, [1240, 2400, 2450]) + ({ LFDNoise3.kr(0.3, 40) } ! 3);
	sw1 = SinOsc.ar(Lag3.kr(sw1_f, 0.01), 0, LFDNoise3.kr(0.3).range(-36, -24).dbamp);
	sw1 = sw1 * Lag3.kr(Trig.kr(sw1_t, sw1_d), 0.01);
	sw1 = PitchShift.ar(sw1, 0.1, 1, 0.005, 0.005);

	sw2_d = Dxrand([0.6] ++ ({ rrand(0.02, 0.09) } ! 15), inf);
	sw2_t = Duty.kr(sw2_d, 0, Dseq([1, 0], inf));
	sw2 = SinOsc.ar(960 + LFDNoise3.kr(0.2, 25), 0, -21.dbamp);
	sw2 = sw2 * Lag3.kr(Trig.kr(sw2_t, TWChoose.kr(sw2_t, [0.04, 0.09], [0.7, 0.3])), 0.005);

	Out.ar(4, Mix([sw1, sw2]) * level);
}).add;
\end{lstlisting}
