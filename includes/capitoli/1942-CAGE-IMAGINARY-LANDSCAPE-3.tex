%!TEX TS-program = xelatex
%!TEX encoding = UTF-8 Unicode
% !TEX root = ../../2017-GS-COME01-INVITO-ASCOLTO.tex

\clearpage

\thispagestyle{empty}

\includepdf[scale=1.05,
		    pagecommand={
		    	\begin{tikzpicture}[
					remember picture,
					overlay]
		    	\node [xshift=2cm,yshift=1cm] at (current page.south west) {\color{white}{\emph{Roberto \textbf{Masotti}}}};
				\end{tikzpicture}}
		    ]{images/masotti/cage.pdf}

\clearpage

%-------------------------------------------------------------
%------------------------- JOHN CAGE - IMAGINARY LANDSCAPE 3 -
%-------------------------------------------------------------

\chapter*{1942. John Cage. \\ \emph{Imaginary Landscape n. 3}.}
\addcontentsline{toc}{chapter}{1942. John Cage. \emph{Imaginary Landscape n. 3}.}

%	\begin{flushright}
%		\textit{Nella nostra anima c'\`e una incrinatura che, se sfiorata, \\
%		risuona come un vaso prezioso riemerso dalle profondit\`a della terra} \\
%		Wassilly Kandinsky - \emph{Lo Spirituale nell'Arte}
%	\end{flushright}
%
%	\begin{flushright}
%		\textit{Music of Changes // John ChAnGEs} \\
%		Pierre Boulez
%	\end{flushright}
%
%	\begin{flushright}
%		\textit{Si dice che i compositori abbiano orecchio per la musica e \\
%		di solito significa che non sentono nulla che arrivi alle loro orecchie. \\
%		Le loro orecchie sono murate dai suoni di loro creazione.} \\
%		John Cage - \emph{45' for a Speaker} (1954)
%	\end{flushright}
%
%\bigskip

Vinile 1

\begin{lstlisting}[style=SuperCollider-IDE]
// Constant Frequency Record
(
SynthDef("constant-frequency", { arg out=0;
    Out.ar(out,
    SinOsc.ar(1000,0, 0.2))
}).play;
s.record(duration:900, numChannels:0);
)
\end{lstlisting}

Vinile 2

\begin{lstlisting}[style=SuperCollider-IDE]
// Continuously Variable Frequency Record
(
SynthDef("continuously-variable-frequency", { arg out=0;
    Out.ar(out,
    SinOsc.ar(LFNoise2.ar(0.5).range(400, 4000), 0, 0.2))
}).play;
s.record(duration:900, numChannels:0);
)
\end{lstlisting}

\clearpage 

Vinile 3

\begin{lstlisting}[style=SuperCollider-IDE]
// Generator Whine Record
(
SynthDef("generator-whine", { arg out=0;
    Out.ar(out,
        LFBrownNoise2.ar(2000, 1, 0, 0.25) + SinOsc.ar(LFNoise2.ar(1234).range(9751, 10000),0, 0.012345)
    )
}).play;
s.record(duration:900, numChannels:0);
)
\end{lstlisting}

