%!TEX TS-program = xelatex
%!TEX encoding = UTF-8 Unicode
% !TEX root = ../../2017-GS-COME01-INVITO-ASCOLTO.tex

\clearpage

\thispagestyle{empty}

\includepdf[scale=1.15,
		    pagecommand={
		    	\begin{tikzpicture}[
					remember picture,
					overlay]
		    	\node [xshift=2cm,yshift=1cm] at (current page.south west) {\color{white}{\emph{Roberto \textbf{Masotti}}}};
				\end{tikzpicture}}
		    ]{images/masotti/cage2.pdf}

\clearpage

%-------------------------------------------------------------
%------------------------------- JOHN CAGE - CARTRIDGE MUSIC -
%-------------------------------------------------------------

\chapter*{1960. John CAGE. \\ \emph{Cartridge Music}.}
\addcontentsline{toc}{chapter}{1960. John CAGE. \emph{Cartridge Music}.}

	\begin{flushright}
		\textit{Nella nostra anima c'\`e una incrinatura che, se sfiorata, \\
		risuona come un vaso prezioso riemerso dalle profondit\`a della terra} \\
		Wassilly Kandinsky - \emph{Lo Spirituale nell'Arte}
	\end{flushright}

	\begin{flushright}
		\textit{Music of Changes // John ChAnGEs} \\
		Pierre Boulez
	\end{flushright}

	\begin{flushright}
		\textit{Si dice che i compositori abbiano orecchio per la musica e \\
		di solito significa che non sentono nulla che arrivi alle loro orecchie. \\
		Le loro orecchie sono murate dai suoni di loro creazione.} \\
		John Cage - \emph{45' for a Speaker} (1954)
	\end{flushright}

\bigskip

%\begin{multicols}{2}

\begin{quote}
	Ogni opera d'arte \`e figlia del suo tempo, e spesso \`e madre dei nostri
	sentimenti.

	Analogamente, ogni periodo culturale esprime una sua arte, che non si ripeter\`a mai pi\`u.
	Lo stesso sforzo di ridar vita a principi estetici del passato pu\`o creare al massimo delle opere d'arte che sembrano bambini nati morti.
	Noi non possiamo, ad esempio, avere la sensibilit\`a e la vita interiore\footnote{Wassilly Kandinsky, \emph{Lo Spirituale nell'Arte}, SE. 1989}\ldots
\end{quote}

di John Cage, in luogo degli antichi Greci, come avrebbe continuato Kandinsky.
Ci deve essere un certo grado di consapevolezza in relazione al livello di
comprensione-incomprensione del pensiero di John Cage. Ma ammettendo di
averlo compreso, per quanto noi potremmo approfondire lo studio del suo
pensiero e della sua musica, potremmo solo arrivare ad imitarne alcuni tratti
stilistici. E se tentassimo di

\begin{quote}
	adottare i loro princ\`{\i}pi, non faremmo che produrre forme simili alle loro, ma prive di anima\footnote{\emph{idem}}.
\end{quote}

Questo non esclude che si possa riuscire ad entrare in contatto con le
motivazioni e gli stimoli artistici, soprattutto
e si attinge ai tratti
condivisi tra le somiglianze delle forme artistiche,

\begin{quote}
	delle aspirazioni interiori e degli ideali (che un tempo erano stati raggiunti
	e poi vennero dimenticati), la somiglianza cio\`e fra i climi culturali di due
	epoche che pu\`o portare alla ripresa di forme che erano gi\`a state utilizzate in
	passato per esprimere le stesse tensioni\footnote{\emph{idem}}.
\end{quote}

Costruire il repertorio, utilizzare il tempo presente per reinventare il tempo passato.
Respirare la stessa aria di un autore non pi\`u presente attraverso la comprensione, intima,
a livello percettivo, delle molteplici ineffabilit\`a della sua epoca.
Consumare la sua poetica acquisendo ogni io sotteso e nascosto. Percepirne il dopo,
il passato meno passato delle conseguenze lasciate al mondo, al suo futuro.
Tutto questo, di un personaggio poliforme ed enorme come John Cage, \`e piuttosto complesso.

%John Cage \`e stato al mondo in maniera totale.

Essere John Cage, \emph{compositore}: ha scritto testi, libri, ha tenuto conferenze, ha dipinto,
ha consultato gli oracoli, ha meditato, ha suonato, ha indagato, ha scritto musica.

%Questo essere, di una sensibilit\`a totale, significa utilizzare la percezione attraverso tutti i canali percettivi di cui si \`e dotati.

In questo senso l'attivit\`a artistica, plastica, sonora, letteraria \`e manifestazione di uno
stesso percepire comune.

Nel luogo temporale di John Cage c'\`e la grande depressione statunitense che trova al suo rientro dall'Europa,
con la quale deve fare i conti, per la prima volta, per guadagnarsi da vivere. Lo fa a suo modo,
sulle tracce di quello che negli anni settanta definir\`a anarchia:

\begin{quote}
	I'm an anarchist. I don't know whether the adjective is pure and simple, or philosophical, or what, but I don't
	like government! And I don't like institutions! And I don't have any confidence in even good institutions.\footnote{John
		Cage at Seventy: An Interview by Stephen Montague. American Music, Summer 1985. Ubu.com. Accessed May 24, 2007.}
\end{quote}

Alla fine degli anni settanta dedica\footnote{\emph{Score (40 Drawings by Thoreau) and 23 Parts} at John Cage's Database of Works www.johncage.org:\\ This work is comprised of drawings by Henry David Thoreau, superimposed on 12 lines, each divided into 5+7+5 segments, the structure of Japanese Haiku poetry (a process similar to that Cage used in the composition of Renga). The tape recording used in this work was made by David Behrman. In performing this work, each individual haiku is to be followed by a silence equal to the length of time of the performance of the This score consists of 20 unnumbered pages plus title page with performance instructions. These 20 pages may be used in whole or in part by between 1 and 20 pianists. The performer(s) make(s) a program of a determined time length and then translates this to the page(s) to be played (with space equating to time). Each page of the score contains 5 systems, notated on 5 bars. Some pages contain very few events, while others are brimming. Most events are aggregates of notes to be played as a single ictus. Dynamics, resonances, overlappings, and interpenetrations are free. Cage?s composing means involved both chance operations and use of the imperfections found in the paper upon which the music was written. This work may be performed with Atlas Eclipticalis or Song Books. This score consists of 20 unnumbered pages plus title page with performance instructions. These 20 pages may be used in whole or in part by between 1 and 20 pianists. The performer(s) make(s) a program of a determined time length and then translates this to the page(s) to be played (with space equating to time). Each page of the score contains 5 systems, notated on 5 bars. Some pages contain very few events, while others are brimming. Most events are aggregates of notes to be played as a single ictus. Dynamics, resonances, overlappings, and interpenetrations are free. Cage?s composing means involved both chance operations and use of the imperfections found in the paper upon which the music was written. This work may be performed with Atlas Eclipticalis or Song Books. All twelve haikus should be followed by the tape recording, equal in duration to performed time of all twelve haiku.}

\begin{quote}
	\emph{Geometria}

	Dentro ogni forma, dietro ogni figura si nasconde una geometria. Questo nascondersi \`e come
	un silenzio che filtra alla superficie con linee sottili e rende intellegibili le forme
	senza che sia necessario comprenderle a fondo: \`e proprio tale muta eloquenza a comunicare
	allo sguardo il senso intimo di un'opera.

	La geometria ha in s\'e elementi ineffabili, pur se immersa in visioni corrusche, in cosmi
	travolti dalle dissimmetrie o capovolti in spirali avvolgenti; e non \`e soltanto
	l'austera evidenza ortogonale di una struttura a farsi Geometria, ma una risonanza segreta
	a far brillare e fiammeggiare, a sollevare e sospendere, ad affermare o togliere al
	corpo della pittura il suo \emph{necessario} silenzio.
\end{quote}

Tra le caratteristiche non convenzionali di John Cage ce n'\`e per me una piuttosto
curiosa: davanti ad una mastodontica produzione musicale, ad una quasi totale
assenza di apparato critico strutturato, non \`e difficile individuare gli
\emph{stili musicali} di John Cage, dividendoli in quattro sezioni temporali
scandite da date piuttosto precise: 1939, 1951, 1969.

Le prime opere sono caratterizzate da un cromatismo strutturato e dalla
sperimentazione soprattutto con gli strumenti a percussione.
Con \emph{Firts Construction in Metal} del 1939 realizza la prima opera
utilizzando strutture temporali. Nel 1951 con \emph{Music of Changes} e
\emph{Imaginary Landscape No. 4} si compie il passaggio dall'organizzazione
sistematica alla combinazione di elementi sistematici, gusti personali e
indeterminazione casuale. Gli anni che seguirono il 1951 furono decisivi per
lo sviluppo delle operazioni casuali.

\`e dal 1957 che Cage inizi\`o a concepire opere nelle quali tutti gli aspetti
dell'interpretazione fossero indeterminati. Tutte le decisioni in merito ai
suoni e alla loro successione sono delegate dal compositore all'esecutore;
la partitura permette soltanto di assicurare una certa disciplina quanto al
modo di prendere decisioni che produrranno dei risultati imprevedibili. In quegli
anni \emph{Fontana Mix, Cartridge Music}\footnote{\emph{Cartridge
Music} at John Cage's Database of Works www.johncage.org: \\ This work was later used as music for the choreographed piece by Merce Cunningham
entitled Changing Steps, with stage and costume design by Charles Atlas (from 1973,
Mark Lancaster); still later, it was used for the choreographed pieces by
Cunningham entitled Exercise Piece II and Exercise Piece III.
The word 'Cartridge' in the title refers to the cartridge of phonographic
pick-ups, into the aperture of which is fitted a needle. In Cartridge Music,
the performer is instructed to insert all manner of unspecified small objects
into the cartridge; prior performances have involved such items as pipe cleaners,
matches, feathers, wires, etc. Furniture may be used as well, amplified via
contact microphones. All sounds are to be amplified and are controlled by the
performer(s). The number of performers should be at least that of the cartridges,
but not greater than twice the number of cartridges. Each performer makes his or
her own part from the materials provided: 20 numbered sheets with irregular
shapes (the number of shapes corresponding to the number of the sheet) and 4
transparencies, one with points, one with circles, another with a circle marked
like a stopwatch, and the last with a dotted curving line, with a circle at one
end. These transparencies are to be superimposed on one of the 20 sheets, in
order to create a constellation from which one creates one's part. It is also
possible to create other pieces from these materials, such as Duet for Cymbal
or a Piano Duet. Cage also used Cartridge Music as a means to compose several
of his lectures, including ?Where Are We Going? And What Are We Doing?? (1960),
?Rhythm, Etc.? (1962), ?Jasper Johns: Stories and Ideas? (1963), and
?On Robert Rauschenberg, Artist, and His Work? (1961).} e la serie delle \emph{Variations},
consisteva in fogli lucidi trasparenti, con linee, punti e curve, a partire dalle
quali si costruivano partiture (strutture, materiali e relazioni, applicabili non
solo alla musica) che era possibile eseguire in diverse circostanze.

La regolarit\`a e la continuit\`a estetica di questa produzione fu interrotta nel 1969
con il brano \emph{HPSCHD} per sette clavicembali amplificati e tape multicanale.
La musica di Cage ri riappropria quindi di una notazione convenzionale in uninione
con la tecnicna del \emph{collage}. L'anno successivo Cage porse le prime domande
di composizione all'\emph{I Ching}.

\begin{quote}
	I procedimenti casuali sono solo uno strumento tra altri che Cage ha utilizzato
	per perseguire coerentemente un unico scopo: l'accettazione disciplinata, in
	contesti musicali, di ci\`o che fino ad allora era stato rifiutato. «Sono sempre
	stato dalla parte delle cose che non si devono fare», ha osservato una volta,
	«cercando il modo di rimettere in gioco gli elementi rifiutati»\footnote{William
	Brooks, \emph{Scelte e Cambiamenti Nella Musica Recente di Cage} 1982}.
\end{quote}

%\centering{***}

\begin{quote}
	L'artista cercher\`a di suscitare sentimenti pi\`u delicati, senza nome [\ldots]
	Attualmente per\`o lo spettatore \`e quasi sempre incapace di emozioni.
	Nell'opera d'arte cerca una mera imitazione della natura a scopo pratico.
	[\ldots] certo l'immedesimazione (e la contrapposizione) non deve essere
	vacua o superficiale: anzi, l'atmosfera dell'opera deve rendere pi\`u
	coinvolgente e visionaria l'atmosfera in cui \`e immerso lo spettatore. [\ldots]
	L'affinamento e la diffusione della loro voce nel tempo e nello spazio
	rimangono per\`o un fatto soggettivo, che non esaurisce le potenzialit\`a dell'arte.
\end{quote}

\begin{quote}
	Indeterminacy gets personal preference out of the compositional process.\footnote{Alvin Lucier, Music 109}
\end{quote}

\begin{quote}
    Cage claimed to be an anarchist. By that he didn't mean that everyone simply does whatever they want to or does things in a shoddy manner. If everybody did whatever they did as well as they could, there wouldn't be the need to refer to a higher authority.
\end{quote}

\begin{quote}
	He wanted to make a composition that was free of personal taste and memory, which existed outside the traditions of music, and was, above all, free of psychology.
There are wonderful images in the \emph{I Ching}? Fire in the Lake, The Wanderer, Inner Truth? but Cage didn't use them.
\end{quote}

\begin{quote}
	La situazione diventava abbastanza confusa, con le persone che ruotavano diverse manopole, senza poter in alcun modo prevedere il risultato delle proprie azioni. (Tom Darter 1982. Lettera a uno sconosciuto)
\end{quote}

\begin{quote}
	\emph{Cartridge Music} prevede che ci siano numerose persone che eseguono programmi che hanno determinato per mezzo di materiali. Ma le azioni di una persona modificheranno in modo non intenzionale le azioni di un'altra, dal momento che le azioni prevedono dei cambiamenti nei controlli di tono e di volume. Cos\`i ci si \`o trovare a suonare qualcosa senza ottenere alcun suono. (Cole Gagne e Tracy Caras 1982)
\end{quote}

\begin{quote}
	\ldots utilizza l'elettronica, ma fa uso anche di oggetti di scarto che sono parte integrante della nostra esistenza quotidiana. Abbiamo una situazione complessa [\ldots] e degli oggetti a cui sono applicati dei \emph{pick-up} [\ldots] Si entra in una situazione simile a quando si imbocca il lungo tunnel del New Jersey. [\ldots] Una persona può magari abbassare il volume, mentre qualcun altro sta suonando qualcosa. Cause ed effetti sono scollegati. Gli elementi personali sembrano far sì che il meccanismo non funzioni in modo proprio perfetto.
\end{quote}

Rapporto con il concerto e la sala da concerto

\begin{quote}
	Se c'è un concerto come questo il pubblico è in grado di sentire anche quando esce dalla sala, e i rumori non sembreranno loro così spiacevoli come avevano pensato. [\ldots] È un tentativo di apriore le nostre menti a possibilità diverse da quelle che possiamo ricordare, e che già sappiamo che ci piacciono. Si deve far qualcosa che ci liberi dai nostri ricordi, dalle nostre preferenze. (David Sterrit 1982)
\end{quote}

\begin{quote}
	\emph{L'atto di formulare domande rappresenta, di fatto, un processo creativo}.\\
	È proprio quello di c ui sono convinto.\\
	\emph{\emph{Cartridge Music} costituisce un buon esempio. in qualche modo è possibile affermare che \emph{Cartridge Music} sempre come \emph{Cartridge Music}. Quel pezzo manterrò la sua identità anche se da un evento all'altro potranno accadere molte cose diverse, e questo a causa dell'invenzione originaria del mezzo attraverso il quale i suoni operano, cioè le testine magnetiche stesse. Questo potrebbe consentirci di affermare che, in una certa misura, ciascun pezzo di John Cage suonerà sempre sia come qualcosa in sé che come un pezzo di John Cage. Pensa questo sia vero?} \\
	Parzialmente.
	\emph{Fino a che punto non è vero?}
	Be, mi fa sempre piacere ricordare quanto avvenne a Beverly Hills mentre prendevo un drink con un'amica, di cui ho dimenticato il nome. Mentre Stavamo parlando e bevendo, c'era un disco che suonava in un'altra stanza, e mi colpì perché si trattava di un pezzo molto interessante. Le chiesi cosa fosse e lei rispose: <<Non stai parlando sul serio vero?>>. Si trattava di un mio pezzo, e non lo avevo affatto riconosciuto.\\
	\emph{E scoprì di che pezzo si trattava?}\\
	Era proprio \emph{Cartridge Music}. Quando david Tudor ed io lo registrammo per la Mainstream, l'etichetta discografica di Earle Brown, Earle ci chiese se desideravamo ascoltare il missaggio finale, e tutti e due rispondemmo che non volevamo sentirlo, e così quella fu probabilmente la prima volta che lo ascoltai. (Tom Darter 1982)

\end{quote}

finale

%\end{multicols}
