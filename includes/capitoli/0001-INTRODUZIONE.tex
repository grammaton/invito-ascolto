%!TEX TS-program = xelatex
%!TEX encoding = UTF-8 Unicode
% !TEX root = ../../2017-GS-COME01-INVITO-ASCOLTO.tex

%-------------------------------------------------------------
%---------------------------------------------- INTRODUZIONE -
%-------------------------------------------------------------

\chapter*{Introduzione}
\addcontentsline{toc}{chapter}{Introduzione}

\vfill

\begin{multicols}{2}

Fare repertorio nel senso pi\`u completo \`e imparare ad ascpltare. Crescere in un processo analitico-conoscitivo che amplifichi le capacità percettive, suonando. In questa direzione diviene superfluo chiedersi \emph{perché fare repertorio} (per imparare ad ascoltare risponderemmo) ma sorge immediatamente la domanda \emph{come?}. Come si fa repertorio. Credo si possa fare repertorio solo ricostruendo, assemblando contesti sulle informazioni disponibili, affinch\'e la pratica poggi su una coscienza, ricostruita, che si stratificherà come pietra calcarea nel sapere sociale. Solo in questo senso il repertorio può, al pari della scrittura, è esercizio, necessità, nutrimento della percezione. Si procede a strati. Dopo aver studiato, letto ed interpretato \emph{Mantra} non si può tornare ad uno strato inferiore di coscienza, ad uno strato precedente di conoscenza. \emph{Mantra} si presenta come un paradigma del sapere musicale. Allo stesso tempo musicale, Cage, illumina la musica, sorride al senso del fare musica, porta l'idea di interprete ad un livello superiore, di manipolazione dell'idea, ai confini della libertà musicale e sociale. Li dove spesso la mancata consapevolezza lascia smarriti e senza anima.

\begin{quote}
	Alla scarsa attenzione alla problematica musicale da parte della riflessione estetica e filosofica in genere, fa riscontro - e alludo qui naturalmente in modo esclusivo alla situazione italiana - nei confronti della questione di una teoria generale della musica, disinteresse che non ha conseguenze solo su maggiori o minori profondità speculative, ma che ha generato una relativa arretratezza nel campo delle indagini più strettamente analitiche che esigono in via di principio opzioni di ordine teorico e metodico spesso apertamente confinanti nell'ambito delle questioni filosofiche [\ldots] La necessità di un punto di vista di una teoria generale si impone qui con particolare evidenza in stretta connessione con problematiche analitiche specifiche e con la consapevolezza del suo raggio di azione che raggiunge il problema della costruzione di un apparato categoriale capace di offrire strumenti per la comprensione delle strutture musicali di culture non europee, così come quello di un  rinnovamento della presentazione dei $\ll$concetti fondamentali$\gg$ che non può non avere conseguenze importanti nella didattica musicale. \\ Giovanni Piana (1991, p. 253, nota 14)
\end{quote}



\end{multicols}
