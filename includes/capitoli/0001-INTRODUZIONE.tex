%!TEX TS-program = xelatex
%!TEX encoding = UTF-8 Unicode
% !TEX root = ../../2017-GS-COME01-INVITO-ASCOLTO.tex

%-------------------------------------------------------------
%---------------------------------------------- INTRODUZIONE -
%-------------------------------------------------------------

\chapter*{Introduzione}
\addcontentsline{toc}{chapter}{Introduzione}

Operando una scelta ci si pongono delle domande, almeno tra gli essere umani. Dovendo scegliere dei brani per il corso di interpretazione ho provato a rispondere ad una domanda piuttosto semplice quanto necessaria: \emph{perché fare repertorio?} Questa domanda, prima di ripsoste, ha generato una serie di successive domande: \emph{cos'è repertorio?} Ed anche rispondendo a queste prime domande \emph{qual è il ruolo del repertorio nel fare contemporaneo?} Si ma \emph{cos'è contemporaneo?}.

Attraverso l'oscuro di questi dubbi ho individuato l'oscuro argomento del mio corso, quindi il focus sul repertorio: l'\emph{ascolto}. \emph{Fare repertorio nell'espressione del suo senso pi\`u completo \`e imparare ad ascoltare.}
 
Dato che le domande sono strettamente intrecciate tra loro, proverei a dare una prima definizione di uomo contemporaneo con le parole di Agamben\index{Agamben, Giorgio}:

\begin{quote}
	Contemporaneo è colui che riceve in pieno viso il fascio di tenebra che proviene dal suo tempo.
\end{quote}

È una definizione fortemente poetica che pone di nuovo la percezione al centro dell'asse uomo-tenebra. La contemporaneità è quindi un momento mobile del tempo che identifica la facoltà di osservare l'oscurità del tempo specifico.

L'osservazione può avvenire solo per distanza perché il contemporaneo, affinando la definizione con le parole di Nietzche, è intempestivo, inattuale.

\begin{quote}
	Appartiene veramente al suo tempo, è veramente contemporaneo colui che non coincide perfettamente con esso, non si adegua alle sue pretese ed è perciò, in questo senzo, inattuale; ma proprio per questo scarto e questo anacronismo, egli è capace pi\`u degli altri di percepire ed afferrare il suo tempo.
\end{quote}

La contemporaneità è quindi

\begin{quote}
	quella relazione col tempo che aderisce a esso attraverso una sfasatura e un anacronismo
\end{quote}

che ci permette di valutare, vedere ed analizzare, alla dovuta distanza. Spazio e tempo quindi legati nell'istante percettivo.

Il tempo del nostro repertorio

\begin{quote}
	è la contemporaneità, esso esige di essere contemporaneo dei testi e degli autori che esamina.
\end{quote}



Attraverso l'oscuro di questi dubbi ho individuato l'oscuro argomento del mio corso, quindi il repertorio, l'\emph{ascolto}: \emph{fare repertorio nell'espressione del suo senso pi\`u completo \`e imparare ad ascoltare.}

Crescere in un processo analitico-conoscitivo che amplifichi le capacità percettive, interpretando, nel significato pratico del termine, praticando. In questa direzione ho reso superfluo chiedersi \emph{perché fare repertorio} (per imparare ad ascoltare risponderemmo) ma al suo posto sorge spontaneamente la domanda \emph{come?} \emph{Come si fa repertorio?} Credo si possa fare repertorio solo ricostruendo, assemblando contesti sulle informazioni disponibili, affinch\'e la pratica poggi su una coscienza, ricostruita, che si stratifichi come pietra calcarea nel sapere sociale. Solo in questo senso il repertorio può essere, al pari della scrittura, esercizio, necessità, nutrimento della percezione.

\begin{quote}
	Alla scarsa attenzione alla problematica musicale da parte della riflessione estetica e filosofica in genere, fa riscontro - e alludo qui naturalmente in modo esclusivo alla situazione italiana - nei confronti della questione di una teoria generale della musica, disinteresse che non ha conseguenze solo su maggiori o minori profondità speculative, ma che ha generato una relativa arretratezza nel campo delle indagini pi\`u strettamente analitiche che esigono in via di principio opzioni di ordine teorico e metodico spesso apertamente confinanti nell'ambito delle questioni filosofiche [\ldots] La necessità di un punto di vista di una teoria generale si impone qui con particolare evidenza in stretta connessione con problematiche analitiche specifiche e con la consapevolezza del suo raggio di azione che raggiunge il problema della costruzione di un apparato categoriale capace di offrire strumenti per la comprensione delle strutture musicali di culture non europee, così come quello di un  rinnovamento della presentazione dei \emph{concetti fondamentali} che non può non avere conseguenze importanti nella didattica musicale. \footnote{Giovanni Piana (1991, p. 253, nota 14)}
\end{quote}

È un procedere che stratifica, solidifica conoscenza. Dopo aver studiato, letto ed interpretato \emph{Mantra} non si può tornare ad uno strato inferiore di coscienza, ad uno strato precedente di conoscenza. \emph{Mantra} si presenta come un paradigma del sapere musicale. Allo stesso tempo musicale, Cage, illumina la musica, sorride al senso del fare musica, porta l'idea di interprete ad un livello superiore, di manipolazione dell'idea, ai confini della libertà musicale e sociale, li dove spesso la mancata consapevolezza lascia smarriti e senza anima.
